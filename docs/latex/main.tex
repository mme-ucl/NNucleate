%% Generated by Sphinx.
\def\sphinxdocclass{report}
\documentclass[a4paper,10pt,english]{report}
\ifdefined\pdfpxdimen
   \let\sphinxpxdimen\pdfpxdimen\else\newdimen\sphinxpxdimen
\fi \sphinxpxdimen=.75bp\relax
\ifdefined\pdfimageresolution
    \pdfimageresolution= \numexpr \dimexpr1in\relax/\sphinxpxdimen\relax
\fi
%% let collapsible pdf bookmarks panel have high depth per default
\PassOptionsToPackage{bookmarksdepth=5}{hyperref}

\PassOptionsToPackage{warn}{textcomp}
\usepackage[utf8]{inputenc}
\ifdefined\DeclareUnicodeCharacter
% support both utf8 and utf8x syntaxes
  \ifdefined\DeclareUnicodeCharacterAsOptional
    \def\sphinxDUC#1{\DeclareUnicodeCharacter{"#1}}
  \else
    \let\sphinxDUC\DeclareUnicodeCharacter
  \fi
  \sphinxDUC{00A0}{\nobreakspace}
  \sphinxDUC{2500}{\sphinxunichar{2500}}
  \sphinxDUC{2502}{\sphinxunichar{2502}}
  \sphinxDUC{2514}{\sphinxunichar{2514}}
  \sphinxDUC{251C}{\sphinxunichar{251C}}
  \sphinxDUC{2572}{\textbackslash}
\fi
\usepackage{cmap}
\usepackage[T1]{fontenc}
\usepackage{amsmath,amssymb,amstext}
\usepackage{babel}


\usepackage{amsmath,amsfonts,amssymb,amsthm}


\usepackage{fncychap}
\usepackage{sphinx}
\sphinxsetup{hmargin={0.7in,0.7in}, vmargin={1in,1in},         verbatimwithframe=true,         TitleColor={rgb}{0,0,0},         HeaderFamily=\rmfamily\bfseries,         InnerLinkColor={rgb}{0,0,1},         OuterLinkColor={rgb}{0,0,1}}
\fvset{fontsize=auto}
\usepackage{geometry}


% Include hyperref last.
\usepackage{hyperref}
% Fix anchor placement for figures with captions.
\usepackage{hypcap}% it must be loaded after hyperref.
% Set up styles of URL: it should be placed after hyperref.
\urlstyle{same}


\usepackage{sphinxmessages}



        %%%%%%%%%%%%%%%%%%%% Meher %%%%%%%%%%%%%%%%%%
        %%%add number to subsubsection 2=subsection, 3=subsubsection
        %%% below subsubsection is not good idea.
        \setcounter{secnumdepth}{3}
        %
        %%%% Table of content upto 2=subsection, 3=subsubsection
        \setcounter{tocdepth}{2}

        \usepackage{amsmath,amsfonts,amssymb,amsthm}
        \usepackage{graphicx}

        %%% reduce spaces for Table of contents, figures and tables
        %%% it is used "\addtocontents{toc}{\vskip -1.2cm}" etc. in the document
        \usepackage[notlot,nottoc,notlof]{}

        \usepackage{color}
        \usepackage{transparent}
        \usepackage{eso-pic}
        \usepackage{lipsum}

        \usepackage{footnotebackref} %%link at the footnote to go to the place of footnote in the text

        %% spacing between line
        \usepackage{setspace}
        %%%%\onehalfspacing
        %%%%\doublespacing
        \singlespacing


        %%%%%%%%%%% datetime
        \usepackage{datetime}

        \newdateformat{MonthYearFormat}{%
            \monthname[\THEMONTH], \THEYEAR}


        %% RO, LE will not work for 'oneside' layout.
        %% Change oneside to twoside in document class
        \usepackage{fancyhdr}
        \pagestyle{fancy}
        \fancyhf{}

        %%% Alternating Header for oneside
        \fancyhead[L]{\ifthenelse{\isodd{\value{page}}}{ \small \nouppercase{\leftmark} }{}}
        \fancyhead[R]{\ifthenelse{\isodd{\value{page}}}{}{ \small \nouppercase{\rightmark} }}

        %%% Alternating Header for two side
        %\fancyhead[RO]{\small \nouppercase{\rightmark}}
        %\fancyhead[LE]{\small \nouppercase{\leftmark}}

        %% for oneside: change footer at right side. If you want to use Left and right then use same as header defined above.
        \fancyfoot[R]{\ifthenelse{\isodd{\value{page}}}{{\tiny Meher Krishna Patel} }{\href{http://pythondsp.readthedocs.io/en/latest/pythondsp/toc.html}{\tiny PythonDSP}}}

        %%% Alternating Footer for two side
        %\fancyfoot[RO, RE]{\scriptsize Meher Krishna Patel (mekrip@gmail.com)}

        %%% page number
        \fancyfoot[CO, CE]{\thepage}

        \renewcommand{\headrulewidth}{0.5pt}
        \renewcommand{\footrulewidth}{0.5pt}

        \RequirePackage{tocbibind} %%% comment this to remove page number for following
        \addto\captionsenglish{\renewcommand{\contentsname}{Table of contents}}
        \addto\captionsenglish{\renewcommand{\listfigurename}{List of figures}}
        \addto\captionsenglish{\renewcommand{\listtablename}{List of tables}}
        % \addto\captionsenglish{\renewcommand{\chaptername}{Chapter}}


        %%reduce spacing for itemize
        \usepackage{enumitem}
        \setlist{nosep}

        %%%%%%%%%%% Quote Styles at the top of chapter
        \usepackage{epigraph}
        \setlength{\epigraphwidth}{0.8\columnwidth}
        \newcommand{\chapterquote}[2]{\epigraphhead[60]{\epigraph{\textit{#1}}{\textbf {\textit{--#2}}}}}
        %%%%%%%%%%% Quote for all places except Chapter
        \newcommand{\sectionquote}[2]{{\quote{\textit{``#1''}}{\textbf {\textit{--#2}}}}}
    

\title{NNucleate documentation}
\date{Sep 06, 2022}
\release{0.0.1}
\author{Florian M. Dietrich}
\newcommand{\sphinxlogo}{\sphinxincludegraphics{logo.jpg}\par}
\renewcommand{\releasename}{ }
\makeindex
\begin{document}

\ifdefined\shorthandoff
  \ifnum\catcode`\=\string=\active\shorthandoff{=}\fi
  \ifnum\catcode`\"=\active\shorthandoff{"}\fi
\fi

\pagestyle{empty}

        \pagenumbering{Roman} %%% to avoid page 1 conflict with actual page 1

        \begin{titlepage}
            \centering

            \vspace*{40mm} %%% * is used to give space from top
            \textbf{\Huge {Sphinx format for Latex and HTML}}

            \vspace{0mm}
            \begin{figure}[!h]
                \centering
                \includegraphics[scale=0.3]{logo.jpg}
            \end{figure}

            \vspace{0mm}
            \Large \textbf{{Meher Krishna Patel}}

            \small Created on : Octorber, 2017

            \vspace*{0mm}
            \small  Last updated : \MonthYearFormat\today


            %% \vfill adds at the bottom
            \vfill
            \small \textit{More documents are freely available at }{\href{http://pythondsp.readthedocs.io/en/latest/pythondsp/toc.html}{PythonDSP}}
        \end{titlepage}

        \clearpage
        \pagenumbering{roman}
        \tableofcontents
        \listoffigures
        \listoftables
        \clearpage
        \pagenumbering{arabic}

        
\pagestyle{plain}
 
\pagestyle{normal}
\phantomsection\label{\detokenize{NNucleate::doc}}



\chapter{Submodules}
\label{\detokenize{NNucleate:submodules}}

\chapter{NNucleate.data\_augmentation module}
\label{\detokenize{NNucleate:module-NNucleate.data_augmentation}}\label{\detokenize{NNucleate:nnucleate-data-augmentation-module}}\index{module@\spxentry{module}!NNucleate.data\_augmentation@\spxentry{NNucleate.data\_augmentation}}\index{NNucleate.data\_augmentation@\spxentry{NNucleate.data\_augmentation}!module@\spxentry{module}}\index{augment\_evenly() (in module NNucleate.data\_augmentation)@\spxentry{augment\_evenly()}\spxextra{in module NNucleate.data\_augmentation}}

\begin{fulllineitems}
\phantomsection\label{\detokenize{NNucleate:NNucleate.data_augmentation.augment_evenly}}
\pysigstartsignatures
\pysiglinewithargsret{\sphinxcode{\sphinxupquote{NNucleate.data\_augmentation.}}\sphinxbfcode{\sphinxupquote{augment\_evenly}}}{\emph{\DUrole{n}{n}\DUrole{p}{:}\DUrole{w}{  }\DUrole{n}{int}}, \emph{\DUrole{n}{trajname}\DUrole{p}{:}\DUrole{w}{  }\DUrole{n}{str}}, \emph{\DUrole{n}{topology}\DUrole{p}{:}\DUrole{w}{  }\DUrole{n}{str}}, \emph{\DUrole{n}{cvname}\DUrole{p}{:}\DUrole{w}{  }\DUrole{n}{str}}, \emph{\DUrole{n}{savename}\DUrole{p}{:}\DUrole{w}{  }\DUrole{n}{str}}, \emph{\DUrole{n}{box}\DUrole{p}{:}\DUrole{w}{  }\DUrole{n}{float}}, \emph{\DUrole{n}{n\_min}\DUrole{o}{=}\DUrole{default_value}{0}}, \emph{\DUrole{n}{col}\DUrole{o}{=}\DUrole{default_value}{3}}, \emph{\DUrole{n}{bins}\DUrole{o}{=}\DUrole{default_value}{25}}, \emph{\DUrole{n}{n\_max}\DUrole{o}{=}\DUrole{default_value}{inf}}}{}
\pysigstopsignatures\begin{description}
\sphinxlineitem{Takes in a trajectory and adds degenerate rotated frames such that the resulting trajectory represents and even histogram.}
\sphinxAtStartPar
Writes a new trajectory and CV file.

\end{description}
\begin{quote}\begin{description}
\sphinxlineitem{Parameters}\begin{itemize}
\item {} 
\sphinxAtStartPar
\sphinxstyleliteralstrong{\sphinxupquote{n}} (\sphinxstyleliteralemphasis{\sphinxupquote{int}}) \textendash{} The height of the target histogram.

\item {} 
\sphinxAtStartPar
\sphinxstyleliteralstrong{\sphinxupquote{trajname}} (\sphinxstyleliteralemphasis{\sphinxupquote{str}}) \textendash{} Path to the trajectory file (.xtc or .xyz).

\item {} 
\sphinxAtStartPar
\sphinxstyleliteralstrong{\sphinxupquote{topology}} (\sphinxstyleliteralemphasis{\sphinxupquote{str}}) \textendash{} Path to the topology file (.pdb).

\item {} 
\sphinxAtStartPar
\sphinxstyleliteralstrong{\sphinxupquote{cvname}} (\sphinxstyleliteralemphasis{\sphinxupquote{str}}) \textendash{} Path to the CV file. Text file with CVs organised in columns.

\item {} 
\sphinxAtStartPar
\sphinxstyleliteralstrong{\sphinxupquote{savename}} (\sphinxstyleliteralemphasis{\sphinxupquote{str}}) \textendash{} String without file ending under which the final CV and traj will be saved.

\item {} 
\sphinxAtStartPar
\sphinxstyleliteralstrong{\sphinxupquote{box}} (\sphinxstyleliteralemphasis{\sphinxupquote{float}}) \textendash{} Box length for applying PBC.

\item {} 
\sphinxAtStartPar
\sphinxstyleliteralstrong{\sphinxupquote{n\_min}} (\sphinxstyleliteralemphasis{\sphinxupquote{int}}\sphinxstyleliteralemphasis{\sphinxupquote{, }}\sphinxstyleliteralemphasis{\sphinxupquote{optional}}) \textendash{} The minimum number of frames to add per frame, defaults to 0.

\item {} 
\sphinxAtStartPar
\sphinxstyleliteralstrong{\sphinxupquote{col}} (\sphinxstyleliteralemphasis{\sphinxupquote{int}}\sphinxstyleliteralemphasis{\sphinxupquote{, }}\sphinxstyleliteralemphasis{\sphinxupquote{optional}}) \textendash{} The column in the CV file from which to read the CV (0 indexing), defaults to 3.

\item {} 
\sphinxAtStartPar
\sphinxstyleliteralstrong{\sphinxupquote{bins}} (\sphinxstyleliteralemphasis{\sphinxupquote{int}}\sphinxstyleliteralemphasis{\sphinxupquote{, }}\sphinxstyleliteralemphasis{\sphinxupquote{optional}}) \textendash{} Number of bins in the target histogram, defaults to 25.

\item {} 
\sphinxAtStartPar
\sphinxstyleliteralstrong{\sphinxupquote{n\_max}} (\sphinxstyleliteralemphasis{\sphinxupquote{int}}\sphinxstyleliteralemphasis{\sphinxupquote{, }}\sphinxstyleliteralemphasis{\sphinxupquote{optional}}) \textendash{} Maximal height of a histogram column, defaults to math.inf.

\end{itemize}

\end{description}\end{quote}

\end{fulllineitems}

\index{transform\_frame\_to\_knn\_list() (in module NNucleate.data\_augmentation)@\spxentry{transform\_frame\_to\_knn\_list()}\spxextra{in module NNucleate.data\_augmentation}}

\begin{fulllineitems}
\phantomsection\label{\detokenize{NNucleate:NNucleate.data_augmentation.transform_frame_to_knn_list}}
\pysigstartsignatures
\pysiglinewithargsret{\sphinxcode{\sphinxupquote{NNucleate.data\_augmentation.}}\sphinxbfcode{\sphinxupquote{transform\_frame\_to\_knn\_list}}}{\emph{\DUrole{n}{k}\DUrole{p}{:}\DUrole{w}{  }\DUrole{n}{int}}, \emph{\DUrole{n}{traj}\DUrole{p}{:}\DUrole{w}{  }\DUrole{n}{ndarray}}, \emph{\DUrole{n}{box\_length}\DUrole{p}{:}\DUrole{w}{  }\DUrole{n}{float}}}{{ $\rightarrow$ ndarray}}
\pysigstopsignatures
\sphinxAtStartPar
Transforms the cartesian representation of a given trajectory frame to a list of sorted distances including the distance of each atom to its k nearest neighbours. This guarantees symmetry invariances but at significant cost and risk of kinks in the CV space.
\begin{quote}\begin{description}
\sphinxlineitem{Parameters}\begin{itemize}
\item {} 
\sphinxAtStartPar
\sphinxstyleliteralstrong{\sphinxupquote{k}} (\sphinxstyleliteralemphasis{\sphinxupquote{int}}) \textendash{} Number of neighbours to consider for each atom.

\item {} 
\sphinxAtStartPar
\sphinxstyleliteralstrong{\sphinxupquote{traj}} (\sphinxstyleliteralemphasis{\sphinxupquote{ndarray of float}}) \textendash{} List of coordinates to be transformed.

\item {} 
\sphinxAtStartPar
\sphinxstyleliteralstrong{\sphinxupquote{box\_length}} (\sphinxstyleliteralemphasis{\sphinxupquote{float}}) \textendash{} Length of the cubic box.

\end{itemize}

\sphinxlineitem{Returns}
\sphinxAtStartPar
Returns an array of shape n\_atoms x k*n\_atoms/2.

\sphinxlineitem{Return type}
\sphinxAtStartPar
ndarray of float

\end{description}\end{quote}

\end{fulllineitems}

\index{transform\_frame\_to\_ndist\_list() (in module NNucleate.data\_augmentation)@\spxentry{transform\_frame\_to\_ndist\_list()}\spxextra{in module NNucleate.data\_augmentation}}

\begin{fulllineitems}
\phantomsection\label{\detokenize{NNucleate:NNucleate.data_augmentation.transform_frame_to_ndist_list}}
\pysigstartsignatures
\pysiglinewithargsret{\sphinxcode{\sphinxupquote{NNucleate.data\_augmentation.}}\sphinxbfcode{\sphinxupquote{transform\_frame\_to\_ndist\_list}}}{\emph{\DUrole{n}{n\_dist}\DUrole{p}{:}\DUrole{w}{  }\DUrole{n}{int}}, \emph{\DUrole{n}{traj}\DUrole{p}{:}\DUrole{w}{  }\DUrole{n}{ndarray}}, \emph{\DUrole{n}{box\_length}\DUrole{p}{:}\DUrole{w}{  }\DUrole{n}{float}}}{{ $\rightarrow$ ndarray}}
\pysigstopsignatures
\sphinxAtStartPar
Transform the the cartesian coordinates of a given trajectory frame into a sorted list of the n\_dist shortest distances in the system.
\begin{quote}\begin{description}
\sphinxlineitem{Parameters}\begin{itemize}
\item {} 
\sphinxAtStartPar
\sphinxstyleliteralstrong{\sphinxupquote{n\_dist}} (\sphinxstyleliteralemphasis{\sphinxupquote{int}}) \textendash{} Number of distances to include (max: n*(n\sphinxhyphen{}1)/2).

\item {} 
\sphinxAtStartPar
\sphinxstyleliteralstrong{\sphinxupquote{traj}} (\sphinxstyleliteralemphasis{\sphinxupquote{ndarray of float}}) \textendash{} List of list of coordinates to transform.

\item {} 
\sphinxAtStartPar
\sphinxstyleliteralstrong{\sphinxupquote{box\_length}} (\sphinxstyleliteralemphasis{\sphinxupquote{float}}) \textendash{} Length of the cubic box.

\end{itemize}

\sphinxlineitem{Returns}
\sphinxAtStartPar
Array of shape n\_atoms x n\_dists.

\sphinxlineitem{Return type}
\sphinxAtStartPar
ndarray of float

\end{description}\end{quote}

\end{fulllineitems}

\index{transform\_traj\_to\_knn\_list() (in module NNucleate.data\_augmentation)@\spxentry{transform\_traj\_to\_knn\_list()}\spxextra{in module NNucleate.data\_augmentation}}

\begin{fulllineitems}
\phantomsection\label{\detokenize{NNucleate:NNucleate.data_augmentation.transform_traj_to_knn_list}}
\pysigstartsignatures
\pysiglinewithargsret{\sphinxcode{\sphinxupquote{NNucleate.data\_augmentation.}}\sphinxbfcode{\sphinxupquote{transform\_traj\_to\_knn\_list}}}{\emph{\DUrole{n}{k}\DUrole{p}{:}\DUrole{w}{  }\DUrole{n}{int}}, \emph{\DUrole{n}{traj}\DUrole{p}{:}\DUrole{w}{  }\DUrole{n}{ndarray}}, \emph{\DUrole{n}{box\_length}\DUrole{p}{:}\DUrole{w}{  }\DUrole{n}{float}}}{{ $\rightarrow$ ndarray}}
\pysigstopsignatures
\sphinxAtStartPar
Transforms the cartesian representation of a given trajectory to a list of sorted distances including the distance of each atom to its k nearest neighbours. This guarantees symmetry invariances but at significant cost and risk of kinks in the CV space.
\begin{quote}\begin{description}
\sphinxlineitem{Parameters}\begin{itemize}
\item {} 
\sphinxAtStartPar
\sphinxstyleliteralstrong{\sphinxupquote{k}} (\sphinxstyleliteralemphasis{\sphinxupquote{int}}) \textendash{} Number of neighbours to consider for each atom.

\item {} 
\sphinxAtStartPar
\sphinxstyleliteralstrong{\sphinxupquote{traj}} (\sphinxstyleliteralemphasis{\sphinxupquote{ndarray of ndarray of float}}) \textendash{} List of coordinates to be transformed.

\item {} 
\sphinxAtStartPar
\sphinxstyleliteralstrong{\sphinxupquote{box\_length}} (\sphinxstyleliteralemphasis{\sphinxupquote{float}}) \textendash{} Length of the cubic box.

\end{itemize}

\sphinxlineitem{Returns}
\sphinxAtStartPar
Returns an array of shape n\_frames x n\_atoms x k*n\_atoms/2.

\sphinxlineitem{Return type}
\sphinxAtStartPar
ndarray of ndarray of float

\end{description}\end{quote}

\end{fulllineitems}

\index{transform\_traj\_to\_ndist\_list() (in module NNucleate.data\_augmentation)@\spxentry{transform\_traj\_to\_ndist\_list()}\spxextra{in module NNucleate.data\_augmentation}}

\begin{fulllineitems}
\phantomsection\label{\detokenize{NNucleate:NNucleate.data_augmentation.transform_traj_to_ndist_list}}
\pysigstartsignatures
\pysiglinewithargsret{\sphinxcode{\sphinxupquote{NNucleate.data\_augmentation.}}\sphinxbfcode{\sphinxupquote{transform\_traj\_to\_ndist\_list}}}{\emph{\DUrole{n}{n\_dist}\DUrole{p}{:}\DUrole{w}{  }\DUrole{n}{int}}, \emph{\DUrole{n}{traj}\DUrole{p}{:}\DUrole{w}{  }\DUrole{n}{ndarray}}, \emph{\DUrole{n}{box\_length}\DUrole{p}{:}\DUrole{w}{  }\DUrole{n}{float}}}{{ $\rightarrow$ ndarray}}
\pysigstopsignatures
\sphinxAtStartPar
Transform the cartesian coordinates of a given trajectory into a sorted list of the n\_dist shortest distances in the system.
\begin{quote}\begin{description}
\sphinxlineitem{Parameters}\begin{itemize}
\item {} 
\sphinxAtStartPar
\sphinxstyleliteralstrong{\sphinxupquote{n\_dist}} (\sphinxstyleliteralemphasis{\sphinxupquote{int}}) \textendash{} Number of distances to include (max: n*(n\sphinxhyphen{}1)/2).

\item {} 
\sphinxAtStartPar
\sphinxstyleliteralstrong{\sphinxupquote{traj}} (\sphinxstyleliteralemphasis{\sphinxupquote{ndarray of ndarray of float}}) \textendash{} Trajectory that is to be transformed.

\item {} 
\sphinxAtStartPar
\sphinxstyleliteralstrong{\sphinxupquote{box\_length}} (\sphinxstyleliteralemphasis{\sphinxupquote{float}}) \textendash{} Length of the cubic box.

\end{itemize}

\sphinxlineitem{Returns}
\sphinxAtStartPar
Array of shape n\_frames x n\_atoms x n\_dists.

\sphinxlineitem{Return type}
\sphinxAtStartPar
ndarray of ndarray of float

\end{description}\end{quote}

\end{fulllineitems}



\chapter{NNucleate.dataset module}
\label{\detokenize{NNucleate:module-NNucleate.dataset}}\label{\detokenize{NNucleate:nnucleate-dataset-module}}\index{module@\spxentry{module}!NNucleate.dataset@\spxentry{NNucleate.dataset}}\index{NNucleate.dataset@\spxentry{NNucleate.dataset}!module@\spxentry{module}}\index{CVTrajectory (class in NNucleate.dataset)@\spxentry{CVTrajectory}\spxextra{class in NNucleate.dataset}}

\begin{fulllineitems}
\phantomsection\label{\detokenize{NNucleate:NNucleate.dataset.CVTrajectory}}
\pysigstartsignatures
\pysiglinewithargsret{\sphinxbfcode{\sphinxupquote{class\DUrole{w}{  }}}\sphinxcode{\sphinxupquote{NNucleate.dataset.}}\sphinxbfcode{\sphinxupquote{CVTrajectory}}}{\emph{\DUrole{n}{cv\_file}\DUrole{p}{:}\DUrole{w}{  }\DUrole{n}{str}}, \emph{\DUrole{n}{traj\_name}\DUrole{p}{:}\DUrole{w}{  }\DUrole{n}{str}}, \emph{\DUrole{n}{top\_file}\DUrole{p}{:}\DUrole{w}{  }\DUrole{n}{str}}, \emph{\DUrole{n}{cv\_col}\DUrole{p}{:}\DUrole{w}{  }\DUrole{n}{int}}, \emph{\DUrole{n}{box\_length}\DUrole{p}{:}\DUrole{w}{  }\DUrole{n}{float}}, \emph{\DUrole{n}{transform}\DUrole{o}{=}\DUrole{default_value}{None}}, \emph{\DUrole{n}{start}\DUrole{o}{=}\DUrole{default_value}{0}}, \emph{\DUrole{n}{stop}\DUrole{o}{=}\DUrole{default_value}{\sphinxhyphen{} 1}}, \emph{\DUrole{n}{stride}\DUrole{o}{=}\DUrole{default_value}{1}}, \emph{\DUrole{n}{root}\DUrole{o}{=}\DUrole{default_value}{1}}}{}
\pysigstopsignatures
\sphinxAtStartPar
Bases: \sphinxcode{\sphinxupquote{Dataset}}

\sphinxAtStartPar
Instantiates a dataset from a trajectory file in xtc/xyz format and a text file containing the nucleation CVs (Assumes cubic cell)

\begin{sphinxadmonition}{warning}{Warning:}
\sphinxAtStartPar
For .xtc give the boxlength in nm and for .xyz give the boxlength in Å.
\end{sphinxadmonition}
\begin{quote}\begin{description}
\sphinxlineitem{Parameters}\begin{itemize}
\item {} 
\sphinxAtStartPar
\sphinxstyleliteralstrong{\sphinxupquote{cv\_file}} (\sphinxstyleliteralemphasis{\sphinxupquote{str}}) \textendash{} Path to text file structured in columns containing the CVs.

\item {} 
\sphinxAtStartPar
\sphinxstyleliteralstrong{\sphinxupquote{traj\_name}} (\sphinxstyleliteralemphasis{\sphinxupquote{str}}) \textendash{} Path to the trajectory in .xtc or .xyz file format.

\item {} 
\sphinxAtStartPar
\sphinxstyleliteralstrong{\sphinxupquote{top\_file}} (\sphinxstyleliteralemphasis{\sphinxupquote{str}}) \textendash{} Path to the topology file in .pdb file format.

\item {} 
\sphinxAtStartPar
\sphinxstyleliteralstrong{\sphinxupquote{cv\_col}} (\sphinxstyleliteralemphasis{\sphinxupquote{int}}) \textendash{} Indicates the column in which the desired CV is written in the CV file (0 indexing).

\item {} 
\sphinxAtStartPar
\sphinxstyleliteralstrong{\sphinxupquote{box\_length}} (\sphinxstyleliteralemphasis{\sphinxupquote{float}}) \textendash{} Length of the cubic cell.

\item {} 
\sphinxAtStartPar
\sphinxstyleliteralstrong{\sphinxupquote{transform}} (\sphinxstyleliteralemphasis{\sphinxupquote{function}}\sphinxstyleliteralemphasis{\sphinxupquote{, }}\sphinxstyleliteralemphasis{\sphinxupquote{optional}}) \textendash{} A function to be applied to the configuration before returning e.g. to\_dist(), defaults to None.

\item {} 
\sphinxAtStartPar
\sphinxstyleliteralstrong{\sphinxupquote{start}} (\sphinxstyleliteralemphasis{\sphinxupquote{int}}\sphinxstyleliteralemphasis{\sphinxupquote{, }}\sphinxstyleliteralemphasis{\sphinxupquote{optional}}) \textendash{} Starting frame of the trajectory, defaults to 0.

\item {} 
\sphinxAtStartPar
\sphinxstyleliteralstrong{\sphinxupquote{stop}} (\sphinxstyleliteralemphasis{\sphinxupquote{int}}\sphinxstyleliteralemphasis{\sphinxupquote{, }}\sphinxstyleliteralemphasis{\sphinxupquote{optional}}) \textendash{} The last file of the trajectory that is read, defaults to \sphinxhyphen{}1.

\item {} 
\sphinxAtStartPar
\sphinxstyleliteralstrong{\sphinxupquote{stride}} (\sphinxstyleliteralemphasis{\sphinxupquote{int}}\sphinxstyleliteralemphasis{\sphinxupquote{, }}\sphinxstyleliteralemphasis{\sphinxupquote{optional}}) \textendash{} The stride with which the trajectory frames are read, defaults to 1.

\item {} 
\sphinxAtStartPar
\sphinxstyleliteralstrong{\sphinxupquote{root}} (\sphinxstyleliteralemphasis{\sphinxupquote{int}}\sphinxstyleliteralemphasis{\sphinxupquote{, }}\sphinxstyleliteralemphasis{\sphinxupquote{optional}}) \textendash{} Allows for the loading of the n\sphinxhyphen{}th root of the CV data (to compress the numerical range), defaults to 1.

\end{itemize}

\end{description}\end{quote}

\end{fulllineitems}

\index{GNNMolecularTrajectory (class in NNucleate.dataset)@\spxentry{GNNMolecularTrajectory}\spxextra{class in NNucleate.dataset}}

\begin{fulllineitems}
\phantomsection\label{\detokenize{NNucleate:NNucleate.dataset.GNNMolecularTrajectory}}
\pysigstartsignatures
\pysiglinewithargsret{\sphinxbfcode{\sphinxupquote{class\DUrole{w}{  }}}\sphinxcode{\sphinxupquote{NNucleate.dataset.}}\sphinxbfcode{\sphinxupquote{GNNMolecularTrajectory}}}{\emph{\DUrole{n}{cv\_file}}, \emph{\DUrole{n}{traj\_name}}, \emph{\DUrole{n}{top\_file}}, \emph{\DUrole{n}{cv\_col}}, \emph{\DUrole{n}{box\_length}}, \emph{\DUrole{n}{rc}}, \emph{\DUrole{n}{n\_mol}}, \emph{\DUrole{n}{n\_at}}, \emph{\DUrole{n}{start}\DUrole{o}{=}\DUrole{default_value}{0}}, \emph{\DUrole{n}{stop}\DUrole{o}{=}\DUrole{default_value}{\sphinxhyphen{} 1}}, \emph{\DUrole{n}{stride}\DUrole{o}{=}\DUrole{default_value}{1}}, \emph{\DUrole{n}{root}\DUrole{o}{=}\DUrole{default_value}{1}}}{}
\pysigstopsignatures
\sphinxAtStartPar
Bases: \sphinxcode{\sphinxupquote{Dataset}}

\sphinxAtStartPar
Generates a dataset from a trajectory in .xtc/.xyz format for the training of a GNN. The edges are generated from the neighbourlist graph between the COMs of the molecules.
\begin{quote}\begin{description}
\sphinxlineitem{Parameters}\begin{itemize}
\item {} 
\sphinxAtStartPar
\sphinxstyleliteralstrong{\sphinxupquote{cv\_file}} (\sphinxstyleliteralemphasis{\sphinxupquote{str}}) \textendash{} Path to the cv file.

\item {} 
\sphinxAtStartPar
\sphinxstyleliteralstrong{\sphinxupquote{traj\_name}} (\sphinxstyleliteralemphasis{\sphinxupquote{str}}) \textendash{} Path to the trajectory file (.xtc/.xyz).

\item {} 
\sphinxAtStartPar
\sphinxstyleliteralstrong{\sphinxupquote{top\_file}} (\sphinxstyleliteralemphasis{\sphinxupquote{str}}) \textendash{} Path to the topology file (.pdb).

\item {} 
\sphinxAtStartPar
\sphinxstyleliteralstrong{\sphinxupquote{cv\_col}} (\sphinxstyleliteralemphasis{\sphinxupquote{int}}) \textendash{} Gives the colimn in which the CV of interest is stored.

\item {} 
\sphinxAtStartPar
\sphinxstyleliteralstrong{\sphinxupquote{box\_length}} (\sphinxstyleliteralemphasis{\sphinxupquote{float}}) \textendash{} Length of the cubic box.

\item {} 
\sphinxAtStartPar
\sphinxstyleliteralstrong{\sphinxupquote{rc}} (\sphinxstyleliteralemphasis{\sphinxupquote{float}}) \textendash{} Cut\sphinxhyphen{}off radius for the construction of the graph.

\item {} 
\sphinxAtStartPar
\sphinxstyleliteralstrong{\sphinxupquote{n\_mol}} (\sphinxstyleliteralemphasis{\sphinxupquote{int}}) \textendash{} Number of molecules in the system

\item {} 
\sphinxAtStartPar
\sphinxstyleliteralstrong{\sphinxupquote{n\_at}} (\sphinxstyleliteralemphasis{\sphinxupquote{int}}) \textendash{} Number of atoms per molecule

\item {} 
\sphinxAtStartPar
\sphinxstyleliteralstrong{\sphinxupquote{start}} (\sphinxstyleliteralemphasis{\sphinxupquote{int}}\sphinxstyleliteralemphasis{\sphinxupquote{, }}\sphinxstyleliteralemphasis{\sphinxupquote{optional}}) \textendash{} Starting frame of the trajectory, defaults to 0.

\item {} 
\sphinxAtStartPar
\sphinxstyleliteralstrong{\sphinxupquote{stop}} (\sphinxstyleliteralemphasis{\sphinxupquote{int}}\sphinxstyleliteralemphasis{\sphinxupquote{, }}\sphinxstyleliteralemphasis{\sphinxupquote{optional}}) \textendash{} The last file of the trajectory that is rea, defaults to \sphinxhyphen{}1.

\item {} 
\sphinxAtStartPar
\sphinxstyleliteralstrong{\sphinxupquote{stride}} (\sphinxstyleliteralemphasis{\sphinxupquote{int}}\sphinxstyleliteralemphasis{\sphinxupquote{, }}\sphinxstyleliteralemphasis{\sphinxupquote{optional}}) \textendash{} The stride with which the trajectory frames are read, defaults to 1.

\item {} 
\sphinxAtStartPar
\sphinxstyleliteralstrong{\sphinxupquote{root}} (\sphinxstyleliteralemphasis{\sphinxupquote{int}}\sphinxstyleliteralemphasis{\sphinxupquote{, }}\sphinxstyleliteralemphasis{\sphinxupquote{optional}}) \textendash{} Allows for the loading of the n\sphinxhyphen{}th root of the CV data (to compress the numerical range), defaults to 1.

\end{itemize}

\end{description}\end{quote}

\end{fulllineitems}

\index{GNNTrajectory (class in NNucleate.dataset)@\spxentry{GNNTrajectory}\spxextra{class in NNucleate.dataset}}

\begin{fulllineitems}
\phantomsection\label{\detokenize{NNucleate:NNucleate.dataset.GNNTrajectory}}
\pysigstartsignatures
\pysiglinewithargsret{\sphinxbfcode{\sphinxupquote{class\DUrole{w}{  }}}\sphinxcode{\sphinxupquote{NNucleate.dataset.}}\sphinxbfcode{\sphinxupquote{GNNTrajectory}}}{\emph{\DUrole{n}{cv\_file}\DUrole{p}{:}\DUrole{w}{  }\DUrole{n}{str}}, \emph{\DUrole{n}{traj\_name}\DUrole{p}{:}\DUrole{w}{  }\DUrole{n}{str}}, \emph{\DUrole{n}{top\_file}\DUrole{p}{:}\DUrole{w}{  }\DUrole{n}{str}}, \emph{\DUrole{n}{cv\_col}\DUrole{p}{:}\DUrole{w}{  }\DUrole{n}{int}}, \emph{\DUrole{n}{box\_length}\DUrole{p}{:}\DUrole{w}{  }\DUrole{n}{float}}, \emph{\DUrole{n}{rc}\DUrole{p}{:}\DUrole{w}{  }\DUrole{n}{float}}, \emph{\DUrole{n}{start}\DUrole{o}{=}\DUrole{default_value}{0}}, \emph{\DUrole{n}{stop}\DUrole{o}{=}\DUrole{default_value}{\sphinxhyphen{} 1}}, \emph{\DUrole{n}{stride}\DUrole{o}{=}\DUrole{default_value}{1}}, \emph{\DUrole{n}{root}\DUrole{o}{=}\DUrole{default_value}{1}}}{}
\pysigstopsignatures
\sphinxAtStartPar
Bases: \sphinxcode{\sphinxupquote{Dataset}}

\sphinxAtStartPar
Generates a dataset from a trajectory in .xtc/.xyz format for the training of a GNN. 
.. warning:: For .xtc give the boxlength in nm and for .xyz give the boxlength in Å.
\begin{quote}\begin{description}
\sphinxlineitem{Parameters}\begin{itemize}
\item {} 
\sphinxAtStartPar
\sphinxstyleliteralstrong{\sphinxupquote{cv\_file}} (\sphinxstyleliteralemphasis{\sphinxupquote{str}}) \textendash{} Path to the cv file.

\item {} 
\sphinxAtStartPar
\sphinxstyleliteralstrong{\sphinxupquote{traj\_name}} (\sphinxstyleliteralemphasis{\sphinxupquote{str}}) \textendash{} Path to the trajectory file (.xtc/.xyz).

\item {} 
\sphinxAtStartPar
\sphinxstyleliteralstrong{\sphinxupquote{top\_file}} (\sphinxstyleliteralemphasis{\sphinxupquote{str}}) \textendash{} Path to the topology file (.pdb).

\item {} 
\sphinxAtStartPar
\sphinxstyleliteralstrong{\sphinxupquote{cv\_col}} (\sphinxstyleliteralemphasis{\sphinxupquote{int}}) \textendash{} Gives the colimn in which the CV of interest is stored.

\item {} 
\sphinxAtStartPar
\sphinxstyleliteralstrong{\sphinxupquote{box\_length}} (\sphinxstyleliteralemphasis{\sphinxupquote{float}}) \textendash{} Length of the cubic box.

\item {} 
\sphinxAtStartPar
\sphinxstyleliteralstrong{\sphinxupquote{rc}} (\sphinxstyleliteralemphasis{\sphinxupquote{float}}) \textendash{} Cut\sphinxhyphen{}off radius for the construction of the graph.

\item {} 
\sphinxAtStartPar
\sphinxstyleliteralstrong{\sphinxupquote{start}} (\sphinxstyleliteralemphasis{\sphinxupquote{int}}\sphinxstyleliteralemphasis{\sphinxupquote{, }}\sphinxstyleliteralemphasis{\sphinxupquote{optional}}) \textendash{} Starting frame of the trajectory, defaults to 0.

\item {} 
\sphinxAtStartPar
\sphinxstyleliteralstrong{\sphinxupquote{stop}} (\sphinxstyleliteralemphasis{\sphinxupquote{int}}\sphinxstyleliteralemphasis{\sphinxupquote{, }}\sphinxstyleliteralemphasis{\sphinxupquote{optional}}) \textendash{} The last file of the trajectory that is rea, defaults to \sphinxhyphen{}1.

\item {} 
\sphinxAtStartPar
\sphinxstyleliteralstrong{\sphinxupquote{stride}} (\sphinxstyleliteralemphasis{\sphinxupquote{int}}\sphinxstyleliteralemphasis{\sphinxupquote{, }}\sphinxstyleliteralemphasis{\sphinxupquote{optional}}) \textendash{} The stride with which the trajectory frames are read, defaults to 1.

\item {} 
\sphinxAtStartPar
\sphinxstyleliteralstrong{\sphinxupquote{root}} (\sphinxstyleliteralemphasis{\sphinxupquote{int}}\sphinxstyleliteralemphasis{\sphinxupquote{, }}\sphinxstyleliteralemphasis{\sphinxupquote{optional}}) \textendash{} Allows for the loading of the n\sphinxhyphen{}th root of the CV data (to compress the numerical range), defaults to 1.

\end{itemize}

\end{description}\end{quote}

\end{fulllineitems}

\index{KNNTrajectory (class in NNucleate.dataset)@\spxentry{KNNTrajectory}\spxextra{class in NNucleate.dataset}}

\begin{fulllineitems}
\phantomsection\label{\detokenize{NNucleate:NNucleate.dataset.KNNTrajectory}}
\pysigstartsignatures
\pysiglinewithargsret{\sphinxbfcode{\sphinxupquote{class\DUrole{w}{  }}}\sphinxcode{\sphinxupquote{NNucleate.dataset.}}\sphinxbfcode{\sphinxupquote{KNNTrajectory}}}{\emph{\DUrole{n}{cv\_file}\DUrole{p}{:}\DUrole{w}{  }\DUrole{n}{str}}, \emph{\DUrole{n}{traj\_name}\DUrole{p}{:}\DUrole{w}{  }\DUrole{n}{str}}, \emph{\DUrole{n}{top\_file}\DUrole{p}{:}\DUrole{w}{  }\DUrole{n}{str}}, \emph{\DUrole{n}{cv\_col}\DUrole{p}{:}\DUrole{w}{  }\DUrole{n}{int}}, \emph{\DUrole{n}{box\_length}\DUrole{p}{:}\DUrole{w}{  }\DUrole{n}{float}}, \emph{\DUrole{n}{k}\DUrole{p}{:}\DUrole{w}{  }\DUrole{n}{int}}, \emph{\DUrole{n}{start}\DUrole{o}{=}\DUrole{default_value}{0}}, \emph{\DUrole{n}{stop}\DUrole{o}{=}\DUrole{default_value}{\sphinxhyphen{} 1}}, \emph{\DUrole{n}{stride}\DUrole{o}{=}\DUrole{default_value}{1}}, \emph{\DUrole{n}{root}\DUrole{o}{=}\DUrole{default_value}{1}}}{}
\pysigstopsignatures
\sphinxAtStartPar
Bases: \sphinxcode{\sphinxupquote{Dataset}}
\begin{description}
\sphinxlineitem{Generates a dataset from a trajectory in .xtc/xyz format. }
\sphinxAtStartPar
The trajectory frames are represented via the sorted distances of all atoms to their k nearest neighbours.

\end{description}

\begin{sphinxadmonition}{warning}{Warning:}
\sphinxAtStartPar
For .xtc give the boxlength in nm and for .xyz give the boxlength in Å.
\end{sphinxadmonition}
\begin{quote}\begin{description}
\sphinxlineitem{Parameters}\begin{itemize}
\item {} 
\sphinxAtStartPar
\sphinxstyleliteralstrong{\sphinxupquote{cv\_file}} (\sphinxstyleliteralemphasis{\sphinxupquote{str}}) \textendash{} Path to the cv file.

\item {} 
\sphinxAtStartPar
\sphinxstyleliteralstrong{\sphinxupquote{traj\_name}} (\sphinxstyleliteralemphasis{\sphinxupquote{str}}) \textendash{} Path to the trajectory file (.xtc/.xyz).

\item {} 
\sphinxAtStartPar
\sphinxstyleliteralstrong{\sphinxupquote{top\_file}} (\sphinxstyleliteralemphasis{\sphinxupquote{str}}) \textendash{} Path to the topology file (.pdb).

\item {} 
\sphinxAtStartPar
\sphinxstyleliteralstrong{\sphinxupquote{cv\_col}} (\sphinxstyleliteralemphasis{\sphinxupquote{int}}) \textendash{} Gives the colimn in which the CV of interest is stored.

\item {} 
\sphinxAtStartPar
\sphinxstyleliteralstrong{\sphinxupquote{box\_length}} (\sphinxstyleliteralemphasis{\sphinxupquote{float}}) \textendash{} Length of the cubic box.

\item {} 
\sphinxAtStartPar
\sphinxstyleliteralstrong{\sphinxupquote{k}} (\sphinxstyleliteralemphasis{\sphinxupquote{int}}) \textendash{} Number of neighbours to consider.

\item {} 
\sphinxAtStartPar
\sphinxstyleliteralstrong{\sphinxupquote{start}} (\sphinxstyleliteralemphasis{\sphinxupquote{int}}\sphinxstyleliteralemphasis{\sphinxupquote{, }}\sphinxstyleliteralemphasis{\sphinxupquote{optional}}) \textendash{} Starting frame of the trajectory, defaults to 0.

\item {} 
\sphinxAtStartPar
\sphinxstyleliteralstrong{\sphinxupquote{stop}} (\sphinxstyleliteralemphasis{\sphinxupquote{int}}\sphinxstyleliteralemphasis{\sphinxupquote{, }}\sphinxstyleliteralemphasis{\sphinxupquote{optional}}) \textendash{} The last file of the trajectory that is read, defaults to \sphinxhyphen{}1.

\item {} 
\sphinxAtStartPar
\sphinxstyleliteralstrong{\sphinxupquote{stride}} (\sphinxstyleliteralemphasis{\sphinxupquote{int}}\sphinxstyleliteralemphasis{\sphinxupquote{, }}\sphinxstyleliteralemphasis{\sphinxupquote{optional}}) \textendash{} The stride with which the trajectory frames are read, defaults to 1.

\item {} 
\sphinxAtStartPar
\sphinxstyleliteralstrong{\sphinxupquote{root}} (\sphinxstyleliteralemphasis{\sphinxupquote{int}}\sphinxstyleliteralemphasis{\sphinxupquote{, }}\sphinxstyleliteralemphasis{\sphinxupquote{optional}}) \textendash{} Allows for the loading of the n\sphinxhyphen{}th root of the CV data (to compress the numerical range), defaults to 1.

\end{itemize}

\end{description}\end{quote}

\end{fulllineitems}

\index{NdistTrajectory (class in NNucleate.dataset)@\spxentry{NdistTrajectory}\spxextra{class in NNucleate.dataset}}

\begin{fulllineitems}
\phantomsection\label{\detokenize{NNucleate:NNucleate.dataset.NdistTrajectory}}
\pysigstartsignatures
\pysiglinewithargsret{\sphinxbfcode{\sphinxupquote{class\DUrole{w}{  }}}\sphinxcode{\sphinxupquote{NNucleate.dataset.}}\sphinxbfcode{\sphinxupquote{NdistTrajectory}}}{\emph{\DUrole{n}{cv\_file}\DUrole{p}{:}\DUrole{w}{  }\DUrole{n}{str}}, \emph{\DUrole{n}{traj\_name}\DUrole{p}{:}\DUrole{w}{  }\DUrole{n}{str}}, \emph{\DUrole{n}{top\_file}\DUrole{p}{:}\DUrole{w}{  }\DUrole{n}{str}}, \emph{\DUrole{n}{cv\_col}\DUrole{p}{:}\DUrole{w}{  }\DUrole{n}{int}}, \emph{\DUrole{n}{box\_length}\DUrole{p}{:}\DUrole{w}{  }\DUrole{n}{float}}, \emph{\DUrole{n}{n\_dist}\DUrole{p}{:}\DUrole{w}{  }\DUrole{n}{int}}, \emph{\DUrole{n}{start}\DUrole{o}{=}\DUrole{default_value}{0}}, \emph{\DUrole{n}{stop}\DUrole{o}{=}\DUrole{default_value}{\sphinxhyphen{} 1}}, \emph{\DUrole{n}{stride}\DUrole{o}{=}\DUrole{default_value}{1}}, \emph{\DUrole{n}{root}\DUrole{o}{=}\DUrole{default_value}{1}}}{}
\pysigstopsignatures
\sphinxAtStartPar
Bases: \sphinxcode{\sphinxupquote{Dataset}}
\begin{description}
\sphinxlineitem{Generates a dataset from a trajectory in .xtc/xyz format. }
\sphinxAtStartPar
The trajectory frames are represented via the n\_dist sorted distances.

\end{description}

\begin{sphinxadmonition}{warning}{Warning:}
\sphinxAtStartPar
For .xtc give the boxlength in nm and for .xyz give the boxlength in Å.
\end{sphinxadmonition}
\begin{quote}\begin{description}
\sphinxlineitem{Parameters}\begin{itemize}
\item {} 
\sphinxAtStartPar
\sphinxstyleliteralstrong{\sphinxupquote{cv\_file}} (\sphinxstyleliteralemphasis{\sphinxupquote{str}}) \textendash{} Path to the cv file.

\item {} 
\sphinxAtStartPar
\sphinxstyleliteralstrong{\sphinxupquote{traj\_name}} (\sphinxstyleliteralemphasis{\sphinxupquote{str}}) \textendash{} Path to the trajectory file (.xtc/.xyz).

\item {} 
\sphinxAtStartPar
\sphinxstyleliteralstrong{\sphinxupquote{top\_file}} (\sphinxstyleliteralemphasis{\sphinxupquote{str}}) \textendash{} Path to the topology file (.pdb).

\item {} 
\sphinxAtStartPar
\sphinxstyleliteralstrong{\sphinxupquote{cv\_col}} (\sphinxstyleliteralemphasis{\sphinxupquote{int}}) \textendash{} Gives the colimn in which the CV of interest is stored.

\item {} 
\sphinxAtStartPar
\sphinxstyleliteralstrong{\sphinxupquote{box\_length}} (\sphinxstyleliteralemphasis{\sphinxupquote{float}}) \textendash{} Length of the cubic box.

\item {} 
\sphinxAtStartPar
\sphinxstyleliteralstrong{\sphinxupquote{n\_dist}} (\sphinxstyleliteralemphasis{\sphinxupquote{int}}) \textendash{} Number of distances to consider.

\item {} 
\sphinxAtStartPar
\sphinxstyleliteralstrong{\sphinxupquote{start}} (\sphinxstyleliteralemphasis{\sphinxupquote{int}}\sphinxstyleliteralemphasis{\sphinxupquote{, }}\sphinxstyleliteralemphasis{\sphinxupquote{optional}}) \textendash{} Starting frame of the trajectory, defaults to 0.

\item {} 
\sphinxAtStartPar
\sphinxstyleliteralstrong{\sphinxupquote{stop}} (\sphinxstyleliteralemphasis{\sphinxupquote{int}}\sphinxstyleliteralemphasis{\sphinxupquote{, }}\sphinxstyleliteralemphasis{\sphinxupquote{optional}}) \textendash{} The last file of the trajectory that is read, defaults to \sphinxhyphen{}1.

\item {} 
\sphinxAtStartPar
\sphinxstyleliteralstrong{\sphinxupquote{stride}} (\sphinxstyleliteralemphasis{\sphinxupquote{int}}\sphinxstyleliteralemphasis{\sphinxupquote{, }}\sphinxstyleliteralemphasis{\sphinxupquote{optional}}) \textendash{} The stride with which the trajectory frames are read, defaults to 1.

\item {} 
\sphinxAtStartPar
\sphinxstyleliteralstrong{\sphinxupquote{root}} (\sphinxstyleliteralemphasis{\sphinxupquote{int}}\sphinxstyleliteralemphasis{\sphinxupquote{, }}\sphinxstyleliteralemphasis{\sphinxupquote{optional}}) \textendash{} Allows for the loading of the n\sphinxhyphen{}th root of the CV data (to compress the numerical range), defaults to 1.

\end{itemize}

\end{description}\end{quote}

\end{fulllineitems}



\chapter{NNucleate.models module}
\label{\detokenize{NNucleate:module-NNucleate.models}}\label{\detokenize{NNucleate:nnucleate-models-module}}\index{module@\spxentry{module}!NNucleate.models@\spxentry{NNucleate.models}}\index{NNucleate.models@\spxentry{NNucleate.models}!module@\spxentry{module}}\index{GCL (class in NNucleate.models)@\spxentry{GCL}\spxextra{class in NNucleate.models}}

\begin{fulllineitems}
\phantomsection\label{\detokenize{NNucleate:NNucleate.models.GCL}}
\pysigstartsignatures
\pysiglinewithargsret{\sphinxbfcode{\sphinxupquote{class\DUrole{w}{  }}}\sphinxcode{\sphinxupquote{NNucleate.models.}}\sphinxbfcode{\sphinxupquote{GCL}}}{\emph{\DUrole{n}{hidden\_nf}\DUrole{p}{:}\DUrole{w}{  }\DUrole{n}{int}}, \emph{\DUrole{n}{act\_fn}\DUrole{o}{=}\DUrole{default_value}{ReLU()}}}{}
\pysigstopsignatures
\sphinxAtStartPar
Bases: \sphinxcode{\sphinxupquote{Module}}

\sphinxAtStartPar
The graph convolutional layer for the graph\sphinxhyphen{}based model. Do not instantiate this directly.
\begin{quote}\begin{description}
\sphinxlineitem{Parameters}\begin{itemize}
\item {} 
\sphinxAtStartPar
\sphinxstyleliteralstrong{\sphinxupquote{hidden\_nf}} (\sphinxstyleliteralemphasis{\sphinxupquote{int}}) \textendash{} Hidden dimensionality of the latent node representation.

\item {} 
\sphinxAtStartPar
\sphinxstyleliteralstrong{\sphinxupquote{act\_fn}} (\sphinxstyleliteralemphasis{\sphinxupquote{torch.nn.modules.activation}}\sphinxstyleliteralemphasis{\sphinxupquote{, }}\sphinxstyleliteralemphasis{\sphinxupquote{optional}}) \textendash{} PyTorch activation function to be used in the multi\sphinxhyphen{}layer perceptrons, defaults to nn.ReLU()

\end{itemize}

\end{description}\end{quote}
\index{edge\_model() (NNucleate.models.GCL method)@\spxentry{edge\_model()}\spxextra{NNucleate.models.GCL method}}

\begin{fulllineitems}
\phantomsection\label{\detokenize{NNucleate:NNucleate.models.GCL.edge_model}}
\pysigstartsignatures
\pysiglinewithargsret{\sphinxbfcode{\sphinxupquote{edge\_model}}}{\emph{\DUrole{n}{source}}, \emph{\DUrole{n}{target}}}{}
\pysigstopsignatures
\end{fulllineitems}

\index{forward() (NNucleate.models.GCL method)@\spxentry{forward()}\spxextra{NNucleate.models.GCL method}}

\begin{fulllineitems}
\phantomsection\label{\detokenize{NNucleate:NNucleate.models.GCL.forward}}
\pysigstartsignatures
\pysiglinewithargsret{\sphinxbfcode{\sphinxupquote{forward}}}{\emph{\DUrole{n}{h}}, \emph{\DUrole{n}{edge\_index}}}{}
\pysigstopsignatures
\sphinxAtStartPar
Defines the computation performed at every call.

\sphinxAtStartPar
Should be overridden by all subclasses.

\begin{sphinxadmonition}{note}{Note:}
\sphinxAtStartPar
Although the recipe for forward pass needs to be defined within
this function, one should call the \sphinxcode{\sphinxupquote{Module}} instance afterwards
instead of this since the former takes care of running the
registered hooks while the latter silently ignores them.
\end{sphinxadmonition}

\end{fulllineitems}

\index{node\_model() (NNucleate.models.GCL method)@\spxentry{node\_model()}\spxextra{NNucleate.models.GCL method}}

\begin{fulllineitems}
\phantomsection\label{\detokenize{NNucleate:NNucleate.models.GCL.node_model}}
\pysigstartsignatures
\pysiglinewithargsret{\sphinxbfcode{\sphinxupquote{node\_model}}}{\emph{\DUrole{n}{x}}, \emph{\DUrole{n}{edge\_index}}, \emph{\DUrole{n}{edge\_attr}}}{}
\pysigstopsignatures
\end{fulllineitems}

\index{training (NNucleate.models.GCL attribute)@\spxentry{training}\spxextra{NNucleate.models.GCL attribute}}

\begin{fulllineitems}
\phantomsection\label{\detokenize{NNucleate:NNucleate.models.GCL.training}}
\pysigstartsignatures
\pysigline{\sphinxbfcode{\sphinxupquote{training}}\sphinxbfcode{\sphinxupquote{\DUrole{p}{:}\DUrole{w}{  }bool}}}
\pysigstopsignatures
\end{fulllineitems}


\end{fulllineitems}

\index{GNNCV (class in NNucleate.models)@\spxentry{GNNCV}\spxextra{class in NNucleate.models}}

\begin{fulllineitems}
\phantomsection\label{\detokenize{NNucleate:NNucleate.models.GNNCV}}
\pysigstartsignatures
\pysiglinewithargsret{\sphinxbfcode{\sphinxupquote{class\DUrole{w}{  }}}\sphinxcode{\sphinxupquote{NNucleate.models.}}\sphinxbfcode{\sphinxupquote{GNNCV}}}{\emph{\DUrole{n}{in\_node\_nf}\DUrole{o}{=}\DUrole{default_value}{3}}, \emph{\DUrole{n}{hidden\_nf}\DUrole{o}{=}\DUrole{default_value}{3}}, \emph{\DUrole{n}{device}\DUrole{o}{=}\DUrole{default_value}{\textquotesingle{}cpu\textquotesingle{}}}, \emph{\DUrole{n}{act\_fn}\DUrole{o}{=}\DUrole{default_value}{ReLU()}}, \emph{\DUrole{n}{n\_layers}\DUrole{o}{=}\DUrole{default_value}{1}}}{}
\pysigstopsignatures
\sphinxAtStartPar
Bases: \sphinxcode{\sphinxupquote{Module}}

\sphinxAtStartPar
\_summary\_
\begin{quote}\begin{description}
\sphinxlineitem{Parameters}\begin{itemize}
\item {} 
\sphinxAtStartPar
\sphinxstyleliteralstrong{\sphinxupquote{in\_node\_nf}} (\sphinxstyleliteralemphasis{\sphinxupquote{int}}\sphinxstyleliteralemphasis{\sphinxupquote{, }}\sphinxstyleliteralemphasis{\sphinxupquote{optional}}) \textendash{} Dimensionality of the data in the graph nodes, defaults to 3.

\item {} 
\sphinxAtStartPar
\sphinxstyleliteralstrong{\sphinxupquote{hidden\_nf}} (\sphinxstyleliteralemphasis{\sphinxupquote{int}}\sphinxstyleliteralemphasis{\sphinxupquote{, }}\sphinxstyleliteralemphasis{\sphinxupquote{optional}}) \textendash{} Hidden dimensionality of the latent node representation, defaults to 3.

\item {} 
\sphinxAtStartPar
\sphinxstyleliteralstrong{\sphinxupquote{device}} (\sphinxstyleliteralemphasis{\sphinxupquote{str}}\sphinxstyleliteralemphasis{\sphinxupquote{, }}\sphinxstyleliteralemphasis{\sphinxupquote{optional}}) \textendash{} Device the model should be stored on (For GPU support), defaults to “cpu”.

\item {} 
\sphinxAtStartPar
\sphinxstyleliteralstrong{\sphinxupquote{act\_fn}} (\sphinxstyleliteralemphasis{\sphinxupquote{torch.nn.modules.activation}}\sphinxstyleliteralemphasis{\sphinxupquote{, }}\sphinxstyleliteralemphasis{\sphinxupquote{optional}}) \textendash{} PyTorch activation function to be used in the multi\sphinxhyphen{}layer perceptrons, defaults to nn.ReLU().

\item {} 
\sphinxAtStartPar
\sphinxstyleliteralstrong{\sphinxupquote{n\_layers}} (\sphinxstyleliteralemphasis{\sphinxupquote{int}}\sphinxstyleliteralemphasis{\sphinxupquote{, }}\sphinxstyleliteralemphasis{\sphinxupquote{optional}}) \textendash{} The number of graph convolutional layers, defaults to 1.

\end{itemize}

\end{description}\end{quote}
\index{forward() (NNucleate.models.GNNCV method)@\spxentry{forward()}\spxextra{NNucleate.models.GNNCV method}}

\begin{fulllineitems}
\phantomsection\label{\detokenize{NNucleate:NNucleate.models.GNNCV.forward}}
\pysigstartsignatures
\pysiglinewithargsret{\sphinxbfcode{\sphinxupquote{forward}}}{\emph{\DUrole{n}{x}}, \emph{\DUrole{n}{edges}}, \emph{\DUrole{n}{n\_nodes}}}{}
\pysigstopsignatures
\sphinxAtStartPar
Defines the computation performed at every call.

\sphinxAtStartPar
Should be overridden by all subclasses.

\begin{sphinxadmonition}{note}{Note:}
\sphinxAtStartPar
Although the recipe for forward pass needs to be defined within
this function, one should call the \sphinxcode{\sphinxupquote{Module}} instance afterwards
instead of this since the former takes care of running the
registered hooks while the latter silently ignores them.
\end{sphinxadmonition}

\end{fulllineitems}

\index{training (NNucleate.models.GNNCV attribute)@\spxentry{training}\spxextra{NNucleate.models.GNNCV attribute}}

\begin{fulllineitems}
\phantomsection\label{\detokenize{NNucleate:NNucleate.models.GNNCV.training}}
\pysigstartsignatures
\pysigline{\sphinxbfcode{\sphinxupquote{training}}\sphinxbfcode{\sphinxupquote{\DUrole{p}{:}\DUrole{w}{  }bool}}}
\pysigstopsignatures
\end{fulllineitems}


\end{fulllineitems}

\index{NNCV (class in NNucleate.models)@\spxentry{NNCV}\spxextra{class in NNucleate.models}}

\begin{fulllineitems}
\phantomsection\label{\detokenize{NNucleate:NNucleate.models.NNCV}}
\pysigstartsignatures
\pysiglinewithargsret{\sphinxbfcode{\sphinxupquote{class\DUrole{w}{  }}}\sphinxcode{\sphinxupquote{NNucleate.models.}}\sphinxbfcode{\sphinxupquote{NNCV}}}{\emph{\DUrole{n}{insize}\DUrole{p}{:}\DUrole{w}{  }\DUrole{n}{int}}, \emph{\DUrole{n}{l1}\DUrole{p}{:}\DUrole{w}{  }\DUrole{n}{int}}, \emph{\DUrole{n}{l2}\DUrole{o}{=}\DUrole{default_value}{0}}, \emph{\DUrole{n}{l3}\DUrole{o}{=}\DUrole{default_value}{0}}}{}
\pysigstopsignatures
\sphinxAtStartPar
Bases: \sphinxcode{\sphinxupquote{Module}}

\sphinxAtStartPar
Instantiates an NN for approximating CVs. Supported are architectures with up to 3 layers.
\begin{quote}\begin{description}
\sphinxlineitem{Parameters}\begin{itemize}
\item {} 
\sphinxAtStartPar
\sphinxstyleliteralstrong{\sphinxupquote{insize}} (\sphinxstyleliteralemphasis{\sphinxupquote{int}}) \textendash{} Size of the input layer.

\item {} 
\sphinxAtStartPar
\sphinxstyleliteralstrong{\sphinxupquote{l1}} (\sphinxstyleliteralemphasis{\sphinxupquote{int}}) \textendash{} Size of dense layer 1.

\item {} 
\sphinxAtStartPar
\sphinxstyleliteralstrong{\sphinxupquote{l2}} (\sphinxstyleliteralemphasis{\sphinxupquote{int}}\sphinxstyleliteralemphasis{\sphinxupquote{, }}\sphinxstyleliteralemphasis{\sphinxupquote{optional}}) \textendash{} Size of dense layer 2, defaults to 0.

\item {} 
\sphinxAtStartPar
\sphinxstyleliteralstrong{\sphinxupquote{l3}} (\sphinxstyleliteralemphasis{\sphinxupquote{int}}\sphinxstyleliteralemphasis{\sphinxupquote{, }}\sphinxstyleliteralemphasis{\sphinxupquote{optional}}) \textendash{} Size of dense layer 3, defaults to 0.

\end{itemize}

\end{description}\end{quote}
\index{forward() (NNucleate.models.NNCV method)@\spxentry{forward()}\spxextra{NNucleate.models.NNCV method}}

\begin{fulllineitems}
\phantomsection\label{\detokenize{NNucleate:NNucleate.models.NNCV.forward}}
\pysigstartsignatures
\pysiglinewithargsret{\sphinxbfcode{\sphinxupquote{forward}}}{\emph{\DUrole{n}{x}}}{}
\pysigstopsignatures
\sphinxAtStartPar
Defines the computation performed at every call.

\sphinxAtStartPar
Should be overridden by all subclasses.

\begin{sphinxadmonition}{note}{Note:}
\sphinxAtStartPar
Although the recipe for forward pass needs to be defined within
this function, one should call the \sphinxcode{\sphinxupquote{Module}} instance afterwards
instead of this since the former takes care of running the
registered hooks while the latter silently ignores them.
\end{sphinxadmonition}

\end{fulllineitems}

\index{training (NNucleate.models.NNCV attribute)@\spxentry{training}\spxextra{NNucleate.models.NNCV attribute}}

\begin{fulllineitems}
\phantomsection\label{\detokenize{NNucleate:NNucleate.models.NNCV.training}}
\pysigstartsignatures
\pysigline{\sphinxbfcode{\sphinxupquote{training}}\sphinxbfcode{\sphinxupquote{\DUrole{p}{:}\DUrole{w}{  }bool}}}
\pysigstopsignatures
\end{fulllineitems}


\end{fulllineitems}

\index{initialise\_weights() (in module NNucleate.models)@\spxentry{initialise\_weights()}\spxextra{in module NNucleate.models}}

\begin{fulllineitems}
\phantomsection\label{\detokenize{NNucleate:NNucleate.models.initialise_weights}}
\pysigstartsignatures
\pysiglinewithargsret{\sphinxcode{\sphinxupquote{NNucleate.models.}}\sphinxbfcode{\sphinxupquote{initialise\_weights}}}{\emph{\DUrole{n}{model}\DUrole{p}{:}\DUrole{w}{  }\DUrole{n}{Module}}}{}
\pysigstopsignatures
\sphinxAtStartPar
Initiallises the weights of a custom model using the globally set seed.
Usage:
model.apply(initialise\_weights)
\begin{quote}\begin{description}
\sphinxlineitem{Parameters}
\sphinxAtStartPar
\sphinxstyleliteralstrong{\sphinxupquote{model}} (\sphinxstyleliteralemphasis{\sphinxupquote{nn.Module}}) \textendash{} Model that is to be initialised

\end{description}\end{quote}

\end{fulllineitems}



\chapter{NNucleate.pycv\_link module}
\label{\detokenize{NNucleate:module-NNucleate.pycv_link}}\label{\detokenize{NNucleate:nnucleate-pycv-link-module}}\index{module@\spxentry{module}!NNucleate.pycv\_link@\spxentry{NNucleate.pycv\_link}}\index{NNucleate.pycv\_link@\spxentry{NNucleate.pycv\_link}!module@\spxentry{module}}\index{write\_cv\_link() (in module NNucleate.pycv\_link)@\spxentry{write\_cv\_link()}\spxextra{in module NNucleate.pycv\_link}}

\begin{fulllineitems}
\phantomsection\label{\detokenize{NNucleate:NNucleate.pycv_link.write_cv_link}}
\pysigstartsignatures
\pysiglinewithargsret{\sphinxcode{\sphinxupquote{NNucleate.pycv\_link.}}\sphinxbfcode{\sphinxupquote{write\_cv\_link}}}{\emph{\DUrole{n}{model}}, \emph{\DUrole{n}{n\_hid}}, \emph{\DUrole{n}{n\_layers}}, \emph{\DUrole{n}{n\_at}}, \emph{\DUrole{n}{box\_l}}, \emph{\DUrole{n}{fname}}}{}
\pysigstopsignatures
\sphinxAtStartPar
Function that writes an input file for the coupling with plumed, based on a model that is passed in the parameters.
This function assumes the following architecture:
\sphinxhyphen{} Embedding layer n\_at x n\_hid
\sphinxhyphen{} N\_layers GCLs
\sphinxhyphen{} Edge layer n\_hid x n\_hid, ReLU, n\_hid x n\_hid
\sphinxhyphen{} Node layer 2*n\_hid x n\_hid, ReLU, n\_hid x n\_hid
\sphinxhyphen{} Node decoder n\_hid x n\_hid, ReLU, n\_hid x n\_hid
\sphinxhyphen{} Graph decoder n\_hid x n\_hid, ReLU, n\_hid x 1
\begin{quote}\begin{description}
\sphinxlineitem{Parameters}\begin{itemize}
\item {} 
\sphinxAtStartPar
\sphinxstyleliteralstrong{\sphinxupquote{model}} ({\hyperref[\detokenize{NNucleate:NNucleate.models.GNNCV}]{\sphinxcrossref{\sphinxstyleliteralemphasis{\sphinxupquote{GNNCV}}}}}) \textendash{} The model for which the input file shall be written. (only for graph\sphinxhyphen{}based models)

\item {} 
\sphinxAtStartPar
\sphinxstyleliteralstrong{\sphinxupquote{n\_hid}} (\sphinxstyleliteralemphasis{\sphinxupquote{int}}) \textendash{} Number of dimensions in the model latent space.

\item {} 
\sphinxAtStartPar
\sphinxstyleliteralstrong{\sphinxupquote{n\_layers}} (\sphinxstyleliteralemphasis{\sphinxupquote{int}}) \textendash{} Number of GCL layers.

\item {} 
\sphinxAtStartPar
\sphinxstyleliteralstrong{\sphinxupquote{n\_at}} (\sphinxstyleliteralemphasis{\sphinxupquote{int}}) \textendash{} Number of nodes in the graph.

\item {} 
\sphinxAtStartPar
\sphinxstyleliteralstrong{\sphinxupquote{box\_l}} (\sphinxstyleliteralemphasis{\sphinxupquote{float}}) \textendash{} Size of the simulation box of the system that is used in the MTD simulation.

\item {} 
\sphinxAtStartPar
\sphinxstyleliteralstrong{\sphinxupquote{fname}} (\sphinxstyleliteralemphasis{\sphinxupquote{str}}) \textendash{} Name of file that is created.

\end{itemize}

\end{description}\end{quote}

\end{fulllineitems}



\chapter{NNucleate.training module}
\label{\detokenize{NNucleate:module-NNucleate.training}}\label{\detokenize{NNucleate:nnucleate-training-module}}\index{module@\spxentry{module}!NNucleate.training@\spxentry{NNucleate.training}}\index{NNucleate.training@\spxentry{NNucleate.training}!module@\spxentry{module}}\index{early\_stopping\_gnn() (in module NNucleate.training)@\spxentry{early\_stopping\_gnn()}\spxextra{in module NNucleate.training}}

\begin{fulllineitems}
\phantomsection\label{\detokenize{NNucleate:NNucleate.training.early_stopping_gnn}}
\pysigstartsignatures
\pysiglinewithargsret{\sphinxcode{\sphinxupquote{NNucleate.training.}}\sphinxbfcode{\sphinxupquote{early\_stopping\_gnn}}}{\emph{\DUrole{n}{model\_t}\DUrole{p}{:}\DUrole{w}{  }\DUrole{n}{{\hyperref[\detokenize{NNucleate:NNucleate.models.GNNCV}]{\sphinxcrossref{GNNCV}}}}}, \emph{\DUrole{n}{train\_loader}\DUrole{p}{:}\DUrole{w}{  }\DUrole{n}{DataLoader}}, \emph{\DUrole{n}{val\_loader}\DUrole{p}{:}\DUrole{w}{  }\DUrole{n}{DataLoader}}, \emph{\DUrole{n}{n\_at}\DUrole{p}{:}\DUrole{w}{  }\DUrole{n}{int}}, \emph{\DUrole{n}{optimizer}\DUrole{p}{:}\DUrole{w}{  }\DUrole{n}{Callable}}, \emph{\DUrole{n}{loss}\DUrole{p}{:}\DUrole{w}{  }\DUrole{n}{Callable}}, \emph{\DUrole{n}{device}\DUrole{p}{:}\DUrole{w}{  }\DUrole{n}{str}}, \emph{\DUrole{n}{test\_freq}\DUrole{o}{=}\DUrole{default_value}{1}}}{{ $\rightarrow$ tuple}}
\pysigstopsignatures
\sphinxAtStartPar
Train a graph\sphinxhyphen{}based model according to the early\sphinxhyphen{}stopping approach.
In early stopping a model is trained until the validation error (approximation for the generalisation error) worsens for the first time to prevent overfitting.
Once an increase in the validation error is detected for the first time th eloop is exited and the model\sphinxhyphen{}state from the \sphinxstyleemphasis{previous} validation is returned.
\begin{quote}\begin{description}
\sphinxlineitem{Parameters}\begin{itemize}
\item {} 
\sphinxAtStartPar
\sphinxstyleliteralstrong{\sphinxupquote{model\_t}} ({\hyperref[\detokenize{NNucleate:NNucleate.models.GNNCV}]{\sphinxcrossref{\sphinxstyleliteralemphasis{\sphinxupquote{GNNCV}}}}}) \textendash{} The graph\sphinxhyphen{}based model hat is to be optimised.

\item {} 
\sphinxAtStartPar
\sphinxstyleliteralstrong{\sphinxupquote{train\_loader}} (\sphinxstyleliteralemphasis{\sphinxupquote{torch.utils.data.Dataloader}}) \textendash{} Wrapper around the training set for the model optimisation.

\item {} 
\sphinxAtStartPar
\sphinxstyleliteralstrong{\sphinxupquote{val\_loader}} (\sphinxstyleliteralemphasis{\sphinxupquote{torch.utils.data.Dataloader}}) \textendash{} Wrapper around the validation set for the model optimisation.

\item {} 
\sphinxAtStartPar
\sphinxstyleliteralstrong{\sphinxupquote{n\_at}} (\sphinxstyleliteralemphasis{\sphinxupquote{int}}) \textendash{} Number of nodes in the graph (Number of atoms or molecules).

\item {} 
\sphinxAtStartPar
\sphinxstyleliteralstrong{\sphinxupquote{optimizer}} (\sphinxstyleliteralemphasis{\sphinxupquote{torch.optim}}) \textendash{} Optimizer to be used for the optimisation.

\item {} 
\sphinxAtStartPar
\sphinxstyleliteralstrong{\sphinxupquote{loss}} (\sphinxstyleliteralemphasis{\sphinxupquote{torch.nn.\_Loss}}) \textendash{} Loss function to be used for the optimisation.

\item {} 
\sphinxAtStartPar
\sphinxstyleliteralstrong{\sphinxupquote{device}} (\sphinxstyleliteralemphasis{\sphinxupquote{str}}) \textendash{} Device that the training is performed on. (Required for GPU compatibility)

\item {} 
\sphinxAtStartPar
\sphinxstyleliteralstrong{\sphinxupquote{test\_freq}} (\sphinxstyleliteralemphasis{\sphinxupquote{int}}\sphinxstyleliteralemphasis{\sphinxupquote{, }}\sphinxstyleliteralemphasis{\sphinxupquote{optional}}) \textendash{} The number of epochs after which the model should be evaluated. A lower number is more accurate and costs more but is reccomended for big datasets, defaults to 1

\end{itemize}

\sphinxlineitem{Returns}
\sphinxAtStartPar
This function returns the optimised model and the history of test and training errors over the course of the convergence.

\sphinxlineitem{Return type}
\sphinxAtStartPar
{\hyperref[\detokenize{NNucleate:NNucleate.models.GNNCV}]{\sphinxcrossref{GNNCV}}}, list of float, list of float

\end{description}\end{quote}

\end{fulllineitems}

\index{evaluate\_model\_gnn() (in module NNucleate.training)@\spxentry{evaluate\_model\_gnn()}\spxextra{in module NNucleate.training}}

\begin{fulllineitems}
\phantomsection\label{\detokenize{NNucleate:NNucleate.training.evaluate_model_gnn}}
\pysigstartsignatures
\pysiglinewithargsret{\sphinxcode{\sphinxupquote{NNucleate.training.}}\sphinxbfcode{\sphinxupquote{evaluate\_model\_gnn}}}{\emph{\DUrole{n}{model}\DUrole{p}{:}\DUrole{w}{  }\DUrole{n}{{\hyperref[\detokenize{NNucleate:NNucleate.models.GNNCV}]{\sphinxcrossref{GNNCV}}}}}, \emph{\DUrole{n}{dataloader}\DUrole{p}{:}\DUrole{w}{  }\DUrole{n}{DataLoader}}, \emph{\DUrole{n}{n\_mol}\DUrole{p}{:}\DUrole{w}{  }\DUrole{n}{int}}, \emph{\DUrole{n}{device}\DUrole{p}{:}\DUrole{w}{  }\DUrole{n}{str}}, \emph{\DUrole{n}{n\_at}\DUrole{o}{=}\DUrole{default_value}{1}}}{{ $\rightarrow$ tuple}}
\pysigstopsignatures
\sphinxAtStartPar
Helper function that evaluates a model on a training set and calculates some properies for the generation of performance scatter plots.
\begin{quote}\begin{description}
\sphinxlineitem{Parameters}\begin{itemize}
\item {} 
\sphinxAtStartPar
\sphinxstyleliteralstrong{\sphinxupquote{model}} ({\hyperref[\detokenize{NNucleate:NNucleate.models.GNNCV}]{\sphinxcrossref{\sphinxstyleliteralemphasis{\sphinxupquote{GNNCV}}}}}) \textendash{} The model that is to be evaluated.

\item {} 
\sphinxAtStartPar
\sphinxstyleliteralstrong{\sphinxupquote{dataloader}} (\sphinxstyleliteralemphasis{\sphinxupquote{torch.utils.data.Dataloader}}) \textendash{} Wrapper around the dataset that the model is supposed to be evaluated on.

\item {} 
\sphinxAtStartPar
\sphinxstyleliteralstrong{\sphinxupquote{n\_mol}} (\sphinxstyleliteralemphasis{\sphinxupquote{int}}) \textendash{} Number of nodes in the graph of each frame. (Number of atoms or molecules)

\item {} 
\sphinxAtStartPar
\sphinxstyleliteralstrong{\sphinxupquote{device}} (\sphinxstyleliteralemphasis{\sphinxupquote{str}}) \textendash{} Device that the training is performed on. (Required for GPU compatibility)

\item {} 
\sphinxAtStartPar
\sphinxstyleliteralstrong{\sphinxupquote{n\_at}} (\sphinxstyleliteralemphasis{\sphinxupquote{int}}\sphinxstyleliteralemphasis{\sphinxupquote{, }}\sphinxstyleliteralemphasis{\sphinxupquote{optional}}) \textendash{} Number of atoms per molecule.

\end{itemize}

\sphinxlineitem{Returns}
\sphinxAtStartPar
Returns the prediction of the model on each frame, the corresponding true values, the root mean square error of the predictions and the r2 correlation coefficient.

\sphinxlineitem{Return type}
\sphinxAtStartPar
List of float, List of float, float, float

\end{description}\end{quote}

\end{fulllineitems}

\index{test\_gnn() (in module NNucleate.training)@\spxentry{test\_gnn()}\spxextra{in module NNucleate.training}}

\begin{fulllineitems}
\phantomsection\label{\detokenize{NNucleate:NNucleate.training.test_gnn}}
\pysigstartsignatures
\pysiglinewithargsret{\sphinxcode{\sphinxupquote{NNucleate.training.}}\sphinxbfcode{\sphinxupquote{test\_gnn}}}{\emph{\DUrole{n}{model}\DUrole{p}{:}\DUrole{w}{  }\DUrole{n}{{\hyperref[\detokenize{NNucleate:NNucleate.models.GNNCV}]{\sphinxcrossref{GNNCV}}}}}, \emph{\DUrole{n}{loader}\DUrole{p}{:}\DUrole{w}{  }\DUrole{n}{DataLoader}}, \emph{\DUrole{n}{n\_mol}\DUrole{p}{:}\DUrole{w}{  }\DUrole{n}{int}}, \emph{\DUrole{n}{loss\_l1}\DUrole{p}{:}\DUrole{w}{  }\DUrole{n}{Callable}}, \emph{\DUrole{n}{device}\DUrole{p}{:}\DUrole{w}{  }\DUrole{n}{str}}, \emph{\DUrole{n}{n\_at}\DUrole{o}{=}\DUrole{default_value}{1}}}{{ $\rightarrow$ float}}
\pysigstopsignatures
\sphinxAtStartPar
Evaluate the test/validation error of a graph based model\_t on a validation set.
\begin{quote}\begin{description}
\sphinxlineitem{Parameters}\begin{itemize}
\item {} 
\sphinxAtStartPar
\sphinxstyleliteralstrong{\sphinxupquote{model}} ({\hyperref[\detokenize{NNucleate:NNucleate.models.GNNCV}]{\sphinxcrossref{\sphinxstyleliteralemphasis{\sphinxupquote{GNNCV}}}}}) \textendash{} Graph\sphinxhyphen{}based model\_t to be trained.

\item {} 
\sphinxAtStartPar
\sphinxstyleliteralstrong{\sphinxupquote{loader}} (\sphinxstyleliteralemphasis{\sphinxupquote{torch.utils.data.Dataloader}}) \textendash{} Wrapper around a GNNTrajectory dataset.

\item {} 
\sphinxAtStartPar
\sphinxstyleliteralstrong{\sphinxupquote{n\_mol}} (\sphinxstyleliteralemphasis{\sphinxupquote{int}}) \textendash{} Number of nodes per frame.

\item {} 
\sphinxAtStartPar
\sphinxstyleliteralstrong{\sphinxupquote{loss\_l1}} (\sphinxstyleliteralemphasis{\sphinxupquote{torch.nn.\_Loss}}) \textendash{} Loss function for the training.

\item {} 
\sphinxAtStartPar
\sphinxstyleliteralstrong{\sphinxupquote{device}} (\sphinxstyleliteralemphasis{\sphinxupquote{str}}) \textendash{} Device that the training is performed on. (Required for GPU compatibility)

\item {} 
\sphinxAtStartPar
\sphinxstyleliteralstrong{\sphinxupquote{n\_at}} (\sphinxstyleliteralemphasis{\sphinxupquote{int}}\sphinxstyleliteralemphasis{\sphinxupquote{, }}\sphinxstyleliteralemphasis{\sphinxupquote{optional}}) \textendash{} Number of atoms per molecule.

\end{itemize}

\sphinxlineitem{Returns}
\sphinxAtStartPar
Return the average loss over the epoch.

\sphinxlineitem{Return type}
\sphinxAtStartPar
float

\end{description}\end{quote}

\end{fulllineitems}

\index{test\_linear() (in module NNucleate.training)@\spxentry{test\_linear()}\spxextra{in module NNucleate.training}}

\begin{fulllineitems}
\phantomsection\label{\detokenize{NNucleate:NNucleate.training.test_linear}}
\pysigstartsignatures
\pysiglinewithargsret{\sphinxcode{\sphinxupquote{NNucleate.training.}}\sphinxbfcode{\sphinxupquote{test\_linear}}}{\emph{\DUrole{n}{model\_t}\DUrole{p}{:}\DUrole{w}{  }\DUrole{n}{{\hyperref[\detokenize{NNucleate:NNucleate.models.NNCV}]{\sphinxcrossref{NNCV}}}}}, \emph{\DUrole{n}{dataloader}\DUrole{p}{:}\DUrole{w}{  }\DUrole{n}{DataLoader}}, \emph{\DUrole{n}{loss\_fn}\DUrole{p}{:}\DUrole{w}{  }\DUrole{n}{Callable}}, \emph{\DUrole{n}{device}\DUrole{p}{:}\DUrole{w}{  }\DUrole{n}{str}}}{{ $\rightarrow$ float}}
\pysigstopsignatures
\sphinxAtStartPar
Calculates the current average test set loss.
\begin{quote}\begin{description}
\sphinxlineitem{Parameters}\begin{itemize}
\item {} 
\sphinxAtStartPar
\sphinxstyleliteralstrong{\sphinxupquote{model\_t}} ({\hyperref[\detokenize{NNucleate:NNucleate.models.NNCV}]{\sphinxcrossref{\sphinxstyleliteralemphasis{\sphinxupquote{NNCV}}}}}) \textendash{} Model that is being trained.

\item {} 
\sphinxAtStartPar
\sphinxstyleliteralstrong{\sphinxupquote{dataloader}} (\sphinxstyleliteralemphasis{\sphinxupquote{torch.utils.data.Dataloader}}) \textendash{} Dataloader loading the test set.

\item {} 
\sphinxAtStartPar
\sphinxstyleliteralstrong{\sphinxupquote{loss\_fn}} (\sphinxstyleliteralemphasis{\sphinxupquote{torch.nn.\_Loss}}) \textendash{} Pytorch loss function.

\item {} 
\sphinxAtStartPar
\sphinxstyleliteralstrong{\sphinxupquote{device}} (\sphinxstyleliteralemphasis{\sphinxupquote{str}}) \textendash{} Device that the training is performed on. (Required for GPU compatibility)

\end{itemize}

\sphinxlineitem{Returns}
\sphinxAtStartPar
Return the validation loss.

\sphinxlineitem{Return type}
\sphinxAtStartPar
float

\end{description}\end{quote}

\end{fulllineitems}

\index{train\_gnn() (in module NNucleate.training)@\spxentry{train\_gnn()}\spxextra{in module NNucleate.training}}

\begin{fulllineitems}
\phantomsection\label{\detokenize{NNucleate:NNucleate.training.train_gnn}}
\pysigstartsignatures
\pysiglinewithargsret{\sphinxcode{\sphinxupquote{NNucleate.training.}}\sphinxbfcode{\sphinxupquote{train\_gnn}}}{\emph{\DUrole{n}{model}\DUrole{p}{:}\DUrole{w}{  }\DUrole{n}{{\hyperref[\detokenize{NNucleate:NNucleate.models.GNNCV}]{\sphinxcrossref{GNNCV}}}}}, \emph{\DUrole{n}{loader}\DUrole{p}{:}\DUrole{w}{  }\DUrole{n}{DataLoader}}, \emph{\DUrole{n}{n\_mol}\DUrole{p}{:}\DUrole{w}{  }\DUrole{n}{int}}, \emph{\DUrole{n}{optimizer}\DUrole{p}{:}\DUrole{w}{  }\DUrole{n}{Callable}}, \emph{\DUrole{n}{loss}\DUrole{p}{:}\DUrole{w}{  }\DUrole{n}{Callable}}, \emph{\DUrole{n}{device}\DUrole{p}{:}\DUrole{w}{  }\DUrole{n}{str}}, \emph{\DUrole{n}{n\_at}\DUrole{o}{=}\DUrole{default_value}{1}}}{{ $\rightarrow$ float}}
\pysigstopsignatures
\sphinxAtStartPar
Function to perform one epoch of a GNN training.
\begin{quote}\begin{description}
\sphinxlineitem{Parameters}\begin{itemize}
\item {} 
\sphinxAtStartPar
\sphinxstyleliteralstrong{\sphinxupquote{model}} ({\hyperref[\detokenize{NNucleate:NNucleate.models.GNNCV}]{\sphinxcrossref{\sphinxstyleliteralemphasis{\sphinxupquote{GNNCV}}}}}) \textendash{} Graph\sphinxhyphen{}based model\_t to be trained.

\item {} 
\sphinxAtStartPar
\sphinxstyleliteralstrong{\sphinxupquote{loader}} (\sphinxstyleliteralemphasis{\sphinxupquote{torch.utils.data.Dataloader}}) \textendash{} Wrapper around a GNNTrajectory dataset.

\item {} 
\sphinxAtStartPar
\sphinxstyleliteralstrong{\sphinxupquote{n\_at}} (\sphinxstyleliteralemphasis{\sphinxupquote{int}}\sphinxstyleliteralemphasis{\sphinxupquote{, }}\sphinxstyleliteralemphasis{\sphinxupquote{optional}}) \textendash{} Number of nodes per frame.

\item {} 
\sphinxAtStartPar
\sphinxstyleliteralstrong{\sphinxupquote{optimizer}} (\sphinxstyleliteralemphasis{\sphinxupquote{torch.optim}}) \textendash{} The optimizer object for the training.

\item {} 
\sphinxAtStartPar
\sphinxstyleliteralstrong{\sphinxupquote{loss}} (\sphinxstyleliteralemphasis{\sphinxupquote{torch.nn.\_Loss}}) \textendash{} Loss function for the training.

\item {} 
\sphinxAtStartPar
\sphinxstyleliteralstrong{\sphinxupquote{device}} (\sphinxstyleliteralemphasis{\sphinxupquote{str}}) \textendash{} Device that the training is performed on. (Required for GPU compatibility)

\item {} 
\sphinxAtStartPar
\sphinxstyleliteralstrong{\sphinxupquote{n\_at}} \textendash{} Number of atoms per molecule.

\end{itemize}

\sphinxlineitem{Returns}
\sphinxAtStartPar
Return the average loss over the epoch.

\sphinxlineitem{Return type}
\sphinxAtStartPar
float

\end{description}\end{quote}

\end{fulllineitems}

\index{train\_linear() (in module NNucleate.training)@\spxentry{train\_linear()}\spxextra{in module NNucleate.training}}

\begin{fulllineitems}
\phantomsection\label{\detokenize{NNucleate:NNucleate.training.train_linear}}
\pysigstartsignatures
\pysiglinewithargsret{\sphinxcode{\sphinxupquote{NNucleate.training.}}\sphinxbfcode{\sphinxupquote{train\_linear}}}{\emph{\DUrole{n}{model\_t}\DUrole{p}{:}\DUrole{w}{  }\DUrole{n}{{\hyperref[\detokenize{NNucleate:NNucleate.models.NNCV}]{\sphinxcrossref{NNCV}}}}}, \emph{\DUrole{n}{dataloader}\DUrole{p}{:}\DUrole{w}{  }\DUrole{n}{DataLoader}}, \emph{\DUrole{n}{loss\_fn}\DUrole{p}{:}\DUrole{w}{  }\DUrole{n}{Callable}}, \emph{\DUrole{n}{optimizer}\DUrole{p}{:}\DUrole{w}{  }\DUrole{n}{Callable}}, \emph{\DUrole{n}{device}\DUrole{p}{:}\DUrole{w}{  }\DUrole{n}{str}}, \emph{\DUrole{n}{print\_batch}\DUrole{o}{=}\DUrole{default_value}{1000000}}}{{ $\rightarrow$ float}}
\pysigstopsignatures
\sphinxAtStartPar
Performs one training epoch for a NNCV.
\begin{quote}\begin{description}
\sphinxlineitem{Parameters}\begin{itemize}
\item {} 
\sphinxAtStartPar
\sphinxstyleliteralstrong{\sphinxupquote{model\_t}} ({\hyperref[\detokenize{NNucleate:NNucleate.models.NNCV}]{\sphinxcrossref{\sphinxstyleliteralemphasis{\sphinxupquote{NNCV}}}}}) \textendash{} The network to be trained.

\item {} 
\sphinxAtStartPar
\sphinxstyleliteralstrong{\sphinxupquote{dataloader}} (\sphinxstyleliteralemphasis{\sphinxupquote{torch.utils.data.Dataloader}}) \textendash{} Wrappper for the training set.

\item {} 
\sphinxAtStartPar
\sphinxstyleliteralstrong{\sphinxupquote{loss\_fn}} (\sphinxstyleliteralemphasis{\sphinxupquote{torch.nn.\_Loss}}) \textendash{} Pytorch loss to be used during training.

\item {} 
\sphinxAtStartPar
\sphinxstyleliteralstrong{\sphinxupquote{optimizer}} (\sphinxstyleliteralemphasis{\sphinxupquote{torch.optim}}) \textendash{} Pytorch optimizer to be used during training.

\item {} 
\sphinxAtStartPar
\sphinxstyleliteralstrong{\sphinxupquote{device}} (\sphinxstyleliteralemphasis{\sphinxupquote{str}}) \textendash{} Pytorch device to run the calculation on. Supports CPU and GPU (cuda).

\item {} 
\sphinxAtStartPar
\sphinxstyleliteralstrong{\sphinxupquote{print\_batch}} (\sphinxstyleliteralemphasis{\sphinxupquote{int}}\sphinxstyleliteralemphasis{\sphinxupquote{, }}\sphinxstyleliteralemphasis{\sphinxupquote{optional}}) \textendash{} Set to recieve printed updates on the lost every print\_batch batches, defaults to 1000000.

\end{itemize}

\sphinxlineitem{Returns}
\sphinxAtStartPar
Returns the last loss item. For easy learning curve recording. Alternatively one can use a Tensorboard.

\sphinxlineitem{Return type}
\sphinxAtStartPar
float

\end{description}\end{quote}

\end{fulllineitems}

\index{train\_perm() (in module NNucleate.training)@\spxentry{train\_perm()}\spxextra{in module NNucleate.training}}

\begin{fulllineitems}
\phantomsection\label{\detokenize{NNucleate:NNucleate.training.train_perm}}
\pysigstartsignatures
\pysiglinewithargsret{\sphinxcode{\sphinxupquote{NNucleate.training.}}\sphinxbfcode{\sphinxupquote{train\_perm}}}{\emph{\DUrole{n}{model\_t}\DUrole{p}{:}\DUrole{w}{  }\DUrole{n}{{\hyperref[\detokenize{NNucleate:NNucleate.models.NNCV}]{\sphinxcrossref{NNCV}}}}}, \emph{\DUrole{n}{dataloader}\DUrole{p}{:}\DUrole{w}{  }\DUrole{n}{DataLoader}}, \emph{\DUrole{n}{optimizer}\DUrole{p}{:}\DUrole{w}{  }\DUrole{n}{Callable}}, \emph{\DUrole{n}{loss\_fn}\DUrole{p}{:}\DUrole{w}{  }\DUrole{n}{Callable}}, \emph{\DUrole{n}{n\_trans}\DUrole{p}{:}\DUrole{w}{  }\DUrole{n}{int}}, \emph{\DUrole{n}{device}\DUrole{p}{:}\DUrole{w}{  }\DUrole{n}{str}}, \emph{\DUrole{n}{print\_batch}\DUrole{o}{=}\DUrole{default_value}{1000000}}}{{ $\rightarrow$ float}}
\pysigstopsignatures
\sphinxAtStartPar
Performs one training epoch for a NNCV but the loss for each batch is not just calculated on one reference structure but a set of n\_trans permutated versions of that structure.
\begin{quote}\begin{description}
\sphinxlineitem{Parameters}\begin{itemize}
\item {} 
\sphinxAtStartPar
\sphinxstyleliteralstrong{\sphinxupquote{dataloader}} (\sphinxstyleliteralemphasis{\sphinxupquote{torch.utils.data.Dataloader}}) \textendash{} Wrapper around a GNNTrajectory dataset.

\item {} 
\sphinxAtStartPar
\sphinxstyleliteralstrong{\sphinxupquote{optimizer}} (\sphinxstyleliteralemphasis{\sphinxupquote{torch.optim}}) \textendash{} The optimizer object for the training.

\item {} 
\sphinxAtStartPar
\sphinxstyleliteralstrong{\sphinxupquote{loss\_fn}} (\sphinxstyleliteralemphasis{\sphinxupquote{torch.nn.\_Loss}}) \textendash{} Loss function for the training.

\item {} 
\sphinxAtStartPar
\sphinxstyleliteralstrong{\sphinxupquote{n\_trans}} (\sphinxstyleliteralemphasis{\sphinxupquote{int}}) \textendash{} Number of permutated structures used for the loss calculations.

\item {} 
\sphinxAtStartPar
\sphinxstyleliteralstrong{\sphinxupquote{device}} (\sphinxstyleliteralemphasis{\sphinxupquote{str}}) \textendash{} Pytorch device to run the calculations on. Supports CPU and GPU (cuda).

\item {} 
\sphinxAtStartPar
\sphinxstyleliteralstrong{\sphinxupquote{print\_batch}} (\sphinxstyleliteralemphasis{\sphinxupquote{int}}\sphinxstyleliteralemphasis{\sphinxupquote{, }}\sphinxstyleliteralemphasis{\sphinxupquote{optional}}) \textendash{} Set to recieve printed updates on the loss every print\_batches batches, defaults to 1000000.

\end{itemize}

\sphinxlineitem{Returns}
\sphinxAtStartPar
Returns the last loss item. For easy learning curve recording. Alternatively one can use a Tensorboard.

\sphinxlineitem{Return type}
\sphinxAtStartPar
float

\end{description}\end{quote}

\end{fulllineitems}

\index{train\_rot() (in module NNucleate.training)@\spxentry{train\_rot()}\spxextra{in module NNucleate.training}}

\begin{fulllineitems}
\phantomsection\label{\detokenize{NNucleate:NNucleate.training.train_rot}}
\pysigstartsignatures
\pysiglinewithargsret{\sphinxcode{\sphinxupquote{NNucleate.training.}}\sphinxbfcode{\sphinxupquote{train\_rot}}}{\emph{\DUrole{n}{model\_t}\DUrole{p}{:}\DUrole{w}{  }\DUrole{n}{{\hyperref[\detokenize{NNucleate:NNucleate.models.NNCV}]{\sphinxcrossref{NNCV}}}}}, \emph{\DUrole{n}{dataloader}\DUrole{p}{:}\DUrole{w}{  }\DUrole{n}{DataLoader}}, \emph{\DUrole{n}{optimizer}\DUrole{p}{:}\DUrole{w}{  }\DUrole{n}{Callable}}, \emph{\DUrole{n}{loss\_fn}\DUrole{p}{:}\DUrole{w}{  }\DUrole{n}{Callable}}, \emph{\DUrole{n}{n\_trans}\DUrole{p}{:}\DUrole{w}{  }\DUrole{n}{int}}, \emph{\DUrole{n}{device}\DUrole{p}{:}\DUrole{w}{  }\DUrole{n}{str}}, \emph{\DUrole{n}{print\_batch}\DUrole{o}{=}\DUrole{default_value}{1000000}}}{{ $\rightarrow$ float}}
\pysigstopsignatures
\sphinxAtStartPar
Performs one training epoch for a NNCV but the loss for each batch is not just calculated on one reference structure but a set of n\_trans rotated versions of that structure.
\begin{quote}\begin{description}
\sphinxlineitem{Parameters}\begin{itemize}
\item {} 
\sphinxAtStartPar
\sphinxstyleliteralstrong{\sphinxupquote{dataloader}} (\sphinxstyleliteralemphasis{\sphinxupquote{torch.utils.data.Dataloader}}) \textendash{} Wrapper around a GNNTrajectory dataset.

\item {} 
\sphinxAtStartPar
\sphinxstyleliteralstrong{\sphinxupquote{optimizer}} (\sphinxstyleliteralemphasis{\sphinxupquote{torch.optim}}) \textendash{} The optimizer object for the training.

\item {} 
\sphinxAtStartPar
\sphinxstyleliteralstrong{\sphinxupquote{loss\_fn}} (\sphinxstyleliteralemphasis{\sphinxupquote{torch.nn.\_Loss}}) \textendash{} Loss function for the training.

\item {} 
\sphinxAtStartPar
\sphinxstyleliteralstrong{\sphinxupquote{n\_trans}} (\sphinxstyleliteralemphasis{\sphinxupquote{int}}) \textendash{} Number of rotated structures used for the loss calculations.

\item {} 
\sphinxAtStartPar
\sphinxstyleliteralstrong{\sphinxupquote{device}} (\sphinxstyleliteralemphasis{\sphinxupquote{str}}) \textendash{} Pytorch device to run the calculations on. Supports CPU and GPU (cuda).

\item {} 
\sphinxAtStartPar
\sphinxstyleliteralstrong{\sphinxupquote{print\_batch}} (\sphinxstyleliteralemphasis{\sphinxupquote{int}}\sphinxstyleliteralemphasis{\sphinxupquote{, }}\sphinxstyleliteralemphasis{\sphinxupquote{optional}}) \textendash{} Set to recieve printed updates on the loss every print\_batches batches, defaults to 1000000.

\end{itemize}

\sphinxlineitem{Returns}
\sphinxAtStartPar
Returns the last loss item. For easy learning curve recording. Alternatively one can use a Tensorboard.

\sphinxlineitem{Return type}
\sphinxAtStartPar
float

\end{description}\end{quote}

\end{fulllineitems}



\chapter{NNucleate.utils module}
\label{\detokenize{NNucleate:module-NNucleate.utils}}\label{\detokenize{NNucleate:nnucleate-utils-module}}\index{module@\spxentry{module}!NNucleate.utils@\spxentry{NNucleate.utils}}\index{NNucleate.utils@\spxentry{NNucleate.utils}!module@\spxentry{module}}\index{PeriodicCKDTree (class in NNucleate.utils)@\spxentry{PeriodicCKDTree}\spxextra{class in NNucleate.utils}}

\begin{fulllineitems}
\phantomsection\label{\detokenize{NNucleate:NNucleate.utils.PeriodicCKDTree}}
\pysigstartsignatures
\pysiglinewithargsret{\sphinxbfcode{\sphinxupquote{class\DUrole{w}{  }}}\sphinxcode{\sphinxupquote{NNucleate.utils.}}\sphinxbfcode{\sphinxupquote{PeriodicCKDTree}}}{\emph{\DUrole{n}{bounds}\DUrole{p}{:}\DUrole{w}{  }\DUrole{n}{ndarray}}, \emph{\DUrole{n}{data}\DUrole{p}{:}\DUrole{w}{  }\DUrole{n}{ndarray}}, \emph{\DUrole{n}{leafsize}\DUrole{o}{=}\DUrole{default_value}{10}}}{}
\pysigstopsignatures
\sphinxAtStartPar
Bases: \sphinxcode{\sphinxupquote{cKDTree}}

\sphinxAtStartPar
A wrapper around scipy.spatial.kdtree to implement periodic boundary conditions

\sphinxAtStartPar
!!!!Written by Patrick Varilly, 6 Jul 2012!!!
“\sphinxurl{https://github.com/patvarilly/periodic\_kdtree}”
Released under the scipy license

\sphinxAtStartPar
Cython kd\sphinxhyphen{}tree for quick nearest\sphinxhyphen{}neighbor lookup with periodic boundaries
See scipy.spatial.ckdtree for details on kd\sphinxhyphen{}trees.
Searches with periodic boundaries are implemented by mapping all
initial data points to one canonical periodic image, building an
ordinary kd\sphinxhyphen{}tree with these points, then querying this kd\sphinxhyphen{}tree multiple
times, if necessary, with all the relevant periodic images of the
query point.
Note that to ensure that no two distinct images of the same point
appear in the results, it is essential to restrict the maximum
distance between a query point and a data point to half the smallest
box dimension.
Construct a kd\sphinxhyphen{}tree.
\begin{quote}\begin{description}
\sphinxlineitem{Parameters}\begin{itemize}
\item {} 
\sphinxAtStartPar
\sphinxstyleliteralstrong{\sphinxupquote{bounds}} (\sphinxstyleliteralemphasis{\sphinxupquote{array\_like}}\sphinxstyleliteralemphasis{\sphinxupquote{, }}\sphinxstyleliteralemphasis{\sphinxupquote{shape}}\sphinxstyleliteralemphasis{\sphinxupquote{ (}}\sphinxstyleliteralemphasis{\sphinxupquote{k}}\sphinxstyleliteralemphasis{\sphinxupquote{,}}\sphinxstyleliteralemphasis{\sphinxupquote{)}}) \textendash{} Size of the periodic box along each spatial dimension.  A
negative or zero size for dimension k means that space is not
periodic along k.

\item {} 
\sphinxAtStartPar
\sphinxstyleliteralstrong{\sphinxupquote{data}} (\sphinxstyleliteralemphasis{\sphinxupquote{array\sphinxhyphen{}like}}\sphinxstyleliteralemphasis{\sphinxupquote{, }}\sphinxstyleliteralemphasis{\sphinxupquote{shape}}\sphinxstyleliteralemphasis{\sphinxupquote{ (}}\sphinxstyleliteralemphasis{\sphinxupquote{n}}\sphinxstyleliteralemphasis{\sphinxupquote{,}}\sphinxstyleliteralemphasis{\sphinxupquote{m}}\sphinxstyleliteralemphasis{\sphinxupquote{)}}) \textendash{} The n data points of dimension mto be indexed. This array is 
not copied unless this is necessary to produce a contiguous 
array of doubles, and so modifying this data will result in 
bogus results.

\item {} 
\sphinxAtStartPar
\sphinxstyleliteralstrong{\sphinxupquote{leafsize}} (\sphinxstyleliteralemphasis{\sphinxupquote{int}}\sphinxstyleliteralemphasis{\sphinxupquote{, }}\sphinxstyleliteralemphasis{\sphinxupquote{optional}}) \textendash{} The number of points at which the algorithm switches over to
brute\sphinxhyphen{}force, defaults to 10.

\end{itemize}

\end{description}\end{quote}
\index{query() (NNucleate.utils.PeriodicCKDTree method)@\spxentry{query()}\spxextra{NNucleate.utils.PeriodicCKDTree method}}

\begin{fulllineitems}
\phantomsection\label{\detokenize{NNucleate:NNucleate.utils.PeriodicCKDTree.query}}
\pysigstartsignatures
\pysiglinewithargsret{\sphinxbfcode{\sphinxupquote{query}}}{\emph{\DUrole{n}{x}\DUrole{p}{:}\DUrole{w}{  }\DUrole{n}{ndarray}}, \emph{\DUrole{n}{k}\DUrole{o}{=}\DUrole{default_value}{1}}, \emph{\DUrole{n}{eps}\DUrole{o}{=}\DUrole{default_value}{0}}, \emph{\DUrole{n}{p}\DUrole{o}{=}\DUrole{default_value}{2}}, \emph{\DUrole{n}{distance\_upper\_bound}\DUrole{o}{=}\DUrole{default_value}{inf}}}{{ $\rightarrow$ ndarray}}
\pysigstopsignatures
\sphinxAtStartPar
Query the kd\sphinxhyphen{}tree for nearest neighbors.
\begin{quote}\begin{description}
\sphinxlineitem{Parameters}\begin{itemize}
\item {} 
\sphinxAtStartPar
\sphinxstyleliteralstrong{\sphinxupquote{x}} (\sphinxstyleliteralemphasis{\sphinxupquote{array\_like}}\sphinxstyleliteralemphasis{\sphinxupquote{, }}\sphinxstyleliteralemphasis{\sphinxupquote{last dimension self.m}}) \textendash{} An array of points to query.

\item {} 
\sphinxAtStartPar
\sphinxstyleliteralstrong{\sphinxupquote{k}} (\sphinxstyleliteralemphasis{\sphinxupquote{int}}\sphinxstyleliteralemphasis{\sphinxupquote{, }}\sphinxstyleliteralemphasis{\sphinxupquote{optional.}}) \textendash{} The number of nearest neighbors to return, defaults to 1

\item {} 
\sphinxAtStartPar
\sphinxstyleliteralstrong{\sphinxupquote{eps}} (\sphinxstyleliteralemphasis{\sphinxupquote{int}}\sphinxstyleliteralemphasis{\sphinxupquote{, }}\sphinxstyleliteralemphasis{\sphinxupquote{optional}}) \textendash{} Return approximate nearest neighbors; the kth returned value 
is guaranteed to be no further than (1+eps) times the 
distance to the real k\sphinxhyphen{}th nearest neighbor, defaults to 0.

\item {} 
\sphinxAtStartPar
\sphinxstyleliteralstrong{\sphinxupquote{p}} (\sphinxstyleliteralemphasis{\sphinxupquote{int}}\sphinxstyleliteralemphasis{\sphinxupquote{, }}\sphinxstyleliteralemphasis{\sphinxupquote{optional}}) \textendash{} Which Minkowski p\sphinxhyphen{}norm to use. 
1 is the sum\sphinxhyphen{}of\sphinxhyphen{}absolute\sphinxhyphen{}values “Manhattan” distance
2 is the usual Euclidean distance
infinity is the maximum\sphinxhyphen{}coordinate\sphinxhyphen{}difference distance, defaults to 2.

\item {} 
\sphinxAtStartPar
\sphinxstyleliteralstrong{\sphinxupquote{distance\_upper\_bound}} (\sphinxstyleliteralemphasis{\sphinxupquote{float}}\sphinxstyleliteralemphasis{\sphinxupquote{, }}\sphinxstyleliteralemphasis{\sphinxupquote{optional}}) \textendash{} Return only neighbors within this distance. This is used to prune
tree searches, so if you are doing a series of nearest\sphinxhyphen{}neighbor
queries, it may help to supply the distance to the nearest neighbor
of the most recent point, defaults to np.inf.

\end{itemize}

\sphinxlineitem{Returns}
\sphinxAtStartPar
The distances to the nearest neighbors. 
If x has shape tuple+(self.m,), then d has shape tuple+(k,).
Missing neighbors are indicated with infinite distances.

\sphinxlineitem{Return type}
\sphinxAtStartPar
array of floats

\sphinxlineitem{Returns}
\sphinxAtStartPar
The locations of the neighbors in self.data.
If \sphinxtitleref{x} has shape tuple+(self.m,), then \sphinxtitleref{i} has shape tuple+(k,).
Missing neighbors are indicated with self.n.

\sphinxlineitem{Return type}
\sphinxAtStartPar
ndarray of ints

\end{description}\end{quote}

\end{fulllineitems}

\index{query\_ball\_point() (NNucleate.utils.PeriodicCKDTree method)@\spxentry{query\_ball\_point()}\spxextra{NNucleate.utils.PeriodicCKDTree method}}

\begin{fulllineitems}
\phantomsection\label{\detokenize{NNucleate:NNucleate.utils.PeriodicCKDTree.query_ball_point}}
\pysigstartsignatures
\pysiglinewithargsret{\sphinxbfcode{\sphinxupquote{query\_ball\_point}}}{\emph{\DUrole{n}{x}\DUrole{p}{:}\DUrole{w}{  }\DUrole{n}{ndarray}}, \emph{\DUrole{n}{r}\DUrole{p}{:}\DUrole{w}{  }\DUrole{n}{float}}, \emph{\DUrole{n}{p}\DUrole{o}{=}\DUrole{default_value}{2.0}}, \emph{\DUrole{n}{eps}\DUrole{o}{=}\DUrole{default_value}{0}}}{{ $\rightarrow$ ndarray}}
\pysigstopsignatures
\sphinxAtStartPar
Find all points within distance r of point(s) x.
Notes: If you have many points whose neighbors you want to find, you may
save substantial amounts of time by putting them in a
PeriodicCKDTree and using query\_ball\_tree.
\begin{quote}\begin{description}
\sphinxlineitem{Parameters}\begin{itemize}
\item {} 
\sphinxAtStartPar
\sphinxstyleliteralstrong{\sphinxupquote{x}} (\sphinxstyleliteralemphasis{\sphinxupquote{array\_like}}\sphinxstyleliteralemphasis{\sphinxupquote{, }}\sphinxstyleliteralemphasis{\sphinxupquote{shape tuple +}}\sphinxstyleliteralemphasis{\sphinxupquote{ (}}\sphinxstyleliteralemphasis{\sphinxupquote{self.m}}\sphinxstyleliteralemphasis{\sphinxupquote{,}}\sphinxstyleliteralemphasis{\sphinxupquote{)}}) \textendash{} The point or points to search for neighbors of.

\item {} 
\sphinxAtStartPar
\sphinxstyleliteralstrong{\sphinxupquote{r}} (\sphinxstyleliteralemphasis{\sphinxupquote{float}}) \textendash{} The radius of points to return.

\item {} 
\sphinxAtStartPar
\sphinxstyleliteralstrong{\sphinxupquote{p}} (\sphinxstyleliteralemphasis{\sphinxupquote{float}}\sphinxstyleliteralemphasis{\sphinxupquote{, }}\sphinxstyleliteralemphasis{\sphinxupquote{optional}}) \textendash{} Which Minkowski p\sphinxhyphen{}norm to use.  Should be in the range {[}1, inf{]}, defaults to 2.0.

\item {} 
\sphinxAtStartPar
\sphinxstyleliteralstrong{\sphinxupquote{eps}} (\sphinxstyleliteralemphasis{\sphinxupquote{int}}\sphinxstyleliteralemphasis{\sphinxupquote{, }}\sphinxstyleliteralemphasis{\sphinxupquote{optional}}) \textendash{} Approximate search. Branches of the tree are not explored if their
nearest points are further than \sphinxcode{\sphinxupquote{r / (1 + eps)}}, and branches are
added in bulk if their furthest points are nearer than
\sphinxcode{\sphinxupquote{r * (1 + eps)}}, defaults to 0.

\end{itemize}

\sphinxlineitem{Returns}
\sphinxAtStartPar
If \sphinxtitleref{x} is a single point, returns a list of the indices of the
neighbors of \sphinxtitleref{x}. If \sphinxtitleref{x} is an array of points, returns an object
array of shape tuple containing lists of neighbors.

\sphinxlineitem{Return type}
\sphinxAtStartPar
list or array of lists

\end{description}\end{quote}

\end{fulllineitems}


\end{fulllineitems}

\index{com() (in module NNucleate.utils)@\spxentry{com()}\spxextra{in module NNucleate.utils}}

\begin{fulllineitems}
\phantomsection\label{\detokenize{NNucleate:NNucleate.utils.com}}
\pysigstartsignatures
\pysiglinewithargsret{\sphinxcode{\sphinxupquote{NNucleate.utils.}}\sphinxbfcode{\sphinxupquote{com}}}{\emph{\DUrole{n}{xyz}\DUrole{p}{:}\DUrole{w}{  }\DUrole{n}{ndarray}}}{{ $\rightarrow$ list}}
\pysigstopsignatures
\sphinxAtStartPar
Calculates the centre of mass of a set of coordinates.
\begin{quote}\begin{description}
\sphinxlineitem{Parameters}
\sphinxAtStartPar
\sphinxstyleliteralstrong{\sphinxupquote{xyz}} (\sphinxstyleliteralemphasis{\sphinxupquote{np.ndarray}}) \textendash{} Array containing the list of 3\sphinxhyphen{}dimensional coordinates.

\sphinxlineitem{Returns}
\sphinxAtStartPar
A list of the calculated centres of mass.

\sphinxlineitem{Return type}
\sphinxAtStartPar
list of float

\end{description}\end{quote}

\end{fulllineitems}

\index{get\_mol\_edges() (in module NNucleate.utils)@\spxentry{get\_mol\_edges()}\spxextra{in module NNucleate.utils}}

\begin{fulllineitems}
\phantomsection\label{\detokenize{NNucleate:NNucleate.utils.get_mol_edges}}
\pysigstartsignatures
\pysiglinewithargsret{\sphinxcode{\sphinxupquote{NNucleate.utils.}}\sphinxbfcode{\sphinxupquote{get\_mol\_edges}}}{\emph{\DUrole{n}{rc}\DUrole{p}{:}\DUrole{w}{  }\DUrole{n}{float}}, \emph{\DUrole{n}{traj}\DUrole{p}{:}\DUrole{w}{  }\DUrole{n}{Trajectory}}, \emph{\DUrole{n}{n\_mol}\DUrole{p}{:}\DUrole{w}{  }\DUrole{n}{int}}, \emph{\DUrole{n}{n\_at}\DUrole{p}{:}\DUrole{w}{  }\DUrole{n}{int}}, \emph{\DUrole{n}{box}\DUrole{p}{:}\DUrole{w}{  }\DUrole{n}{float}}}{{ $\rightarrow$ list}}
\pysigstopsignatures
\sphinxAtStartPar
Generate the edges for a neighbourlist graph based on the COMs of the given molecules.
\begin{quote}\begin{description}
\sphinxlineitem{Parameters}\begin{itemize}
\item {} 
\sphinxAtStartPar
\sphinxstyleliteralstrong{\sphinxupquote{rc}} (\sphinxstyleliteralemphasis{\sphinxupquote{float}}) \textendash{} Cut off radius for the neighbourlist graph.

\item {} 
\sphinxAtStartPar
\sphinxstyleliteralstrong{\sphinxupquote{traj}} (\sphinxstyleliteralemphasis{\sphinxupquote{md.Trajectory}}) \textendash{} The Trajectory containing the frames.

\item {} 
\sphinxAtStartPar
\sphinxstyleliteralstrong{\sphinxupquote{n\_mol}} (\sphinxstyleliteralemphasis{\sphinxupquote{int}}) \textendash{} Number of molecules per frame.

\item {} 
\sphinxAtStartPar
\sphinxstyleliteralstrong{\sphinxupquote{n\_at}} (\sphinxstyleliteralemphasis{\sphinxupquote{int}}) \textendash{} Number of atoms per molecule.

\item {} 
\sphinxAtStartPar
\sphinxstyleliteralstrong{\sphinxupquote{box}} (\sphinxstyleliteralemphasis{\sphinxupquote{float}}) \textendash{} Length of the cubic box

\end{itemize}

\sphinxlineitem{Returns}
\sphinxAtStartPar
A list containing two tensors which represent the adjacency matrix of the graph.

\sphinxlineitem{Return type}
\sphinxAtStartPar
list of torch.tensor

\end{description}\end{quote}

\end{fulllineitems}

\index{get\_rc\_edges() (in module NNucleate.utils)@\spxentry{get\_rc\_edges()}\spxextra{in module NNucleate.utils}}

\begin{fulllineitems}
\phantomsection\label{\detokenize{NNucleate:NNucleate.utils.get_rc_edges}}
\pysigstartsignatures
\pysiglinewithargsret{\sphinxcode{\sphinxupquote{NNucleate.utils.}}\sphinxbfcode{\sphinxupquote{get\_rc\_edges}}}{\emph{\DUrole{n}{rc}\DUrole{p}{:}\DUrole{w}{  }\DUrole{n}{float}}, \emph{\DUrole{n}{traj}\DUrole{p}{:}\DUrole{w}{  }\DUrole{n}{Trajectory}}}{{ $\rightarrow$ list}}
\pysigstopsignatures
\sphinxAtStartPar
Returns the edges of the graph constructed by interpreting the atoms in the trajectory as nodes that are connected to all other nodes within a distance of rc.
\begin{quote}\begin{description}
\sphinxlineitem{Parameters}\begin{itemize}
\item {} 
\sphinxAtStartPar
\sphinxstyleliteralstrong{\sphinxupquote{rc}} (\sphinxstyleliteralemphasis{\sphinxupquote{float}}) \textendash{} Cut\sphinxhyphen{}off radius for the graph construction.

\item {} 
\sphinxAtStartPar
\sphinxstyleliteralstrong{\sphinxupquote{traj}} (\sphinxstyleliteralemphasis{\sphinxupquote{md.trajectory}}) \textendash{} The trajectory for which the graphs shall be constructed.

\end{itemize}

\sphinxlineitem{Returns}
\sphinxAtStartPar
A list containing two tensors which represent the adjacency matrix of the graph.

\sphinxlineitem{Return type}
\sphinxAtStartPar
list of torch.tensor

\end{description}\end{quote}

\end{fulllineitems}

\index{pbc() (in module NNucleate.utils)@\spxentry{pbc()}\spxextra{in module NNucleate.utils}}

\begin{fulllineitems}
\phantomsection\label{\detokenize{NNucleate:NNucleate.utils.pbc}}
\pysigstartsignatures
\pysiglinewithargsret{\sphinxcode{\sphinxupquote{NNucleate.utils.}}\sphinxbfcode{\sphinxupquote{pbc}}}{\emph{\DUrole{n}{trajectory}\DUrole{p}{:}\DUrole{w}{  }\DUrole{n}{Trajectory}}, \emph{\DUrole{n}{box\_length}\DUrole{p}{:}\DUrole{w}{  }\DUrole{n}{float}}}{{ $\rightarrow$ Trajectory}}
\pysigstopsignatures
\sphinxAtStartPar
Centers an mdtraj Trajectory around the centre of a cubic box with the given box length and wraps all atoms into the box.
\begin{quote}\begin{description}
\sphinxlineitem{Parameters}\begin{itemize}
\item {} 
\sphinxAtStartPar
\sphinxstyleliteralstrong{\sphinxupquote{trajectory}} (\sphinxstyleliteralemphasis{\sphinxupquote{mdtraj.trajectory}}) \textendash{} The trajectory that is to be modified, i.e. contains the configurations that shall be wrapped back into the simulation box.

\item {} 
\sphinxAtStartPar
\sphinxstyleliteralstrong{\sphinxupquote{box\_length}} (\sphinxstyleliteralemphasis{\sphinxupquote{float}}) \textendash{} Length of the cubic box which shall contain all the positions.

\end{itemize}

\sphinxlineitem{Returns}
\sphinxAtStartPar
Returns a trajectory object obeying PBC according to the given box length.

\sphinxlineitem{Return type}
\sphinxAtStartPar
mdtraj.trajectory

\end{description}\end{quote}

\end{fulllineitems}

\index{pbc\_config() (in module NNucleate.utils)@\spxentry{pbc\_config()}\spxextra{in module NNucleate.utils}}

\begin{fulllineitems}
\phantomsection\label{\detokenize{NNucleate:NNucleate.utils.pbc_config}}
\pysigstartsignatures
\pysiglinewithargsret{\sphinxcode{\sphinxupquote{NNucleate.utils.}}\sphinxbfcode{\sphinxupquote{pbc\_config}}}{\emph{\DUrole{n}{config}\DUrole{p}{:}\DUrole{w}{  }\DUrole{n}{ndarray}}, \emph{\DUrole{n}{box\_length}\DUrole{p}{:}\DUrole{w}{  }\DUrole{n}{float}}}{{ $\rightarrow$ Trajectory}}
\pysigstopsignatures
\sphinxAtStartPar
Wraps all atoms in a given configuration into the box.
\begin{quote}\begin{description}
\sphinxlineitem{Parameters}\begin{itemize}
\item {} 
\sphinxAtStartPar
\sphinxstyleliteralstrong{\sphinxupquote{config}} \textendash{} The trajectory that is to be modified, i.e. contains the configurations that shall be wrapped back into the simulation box.

\item {} 
\sphinxAtStartPar
\sphinxstyleliteralstrong{\sphinxupquote{box\_length}} (\sphinxstyleliteralemphasis{\sphinxupquote{float}}) \textendash{} Length of the cubic box which shall contain all the positions.

\end{itemize}

\sphinxlineitem{Returns}
\sphinxAtStartPar
Returns a trajectory object obeying PBC according to the given box length.

\sphinxlineitem{Return type}
\sphinxAtStartPar
np.ndarray

\end{description}\end{quote}

\end{fulllineitems}

\index{rotate\_trajs() (in module NNucleate.utils)@\spxentry{rotate\_trajs()}\spxextra{in module NNucleate.utils}}

\begin{fulllineitems}
\phantomsection\label{\detokenize{NNucleate:NNucleate.utils.rotate_trajs}}
\pysigstartsignatures
\pysiglinewithargsret{\sphinxcode{\sphinxupquote{NNucleate.utils.}}\sphinxbfcode{\sphinxupquote{rotate\_trajs}}}{\emph{\DUrole{n}{trajectories}\DUrole{p}{:}\DUrole{w}{  }\DUrole{n}{ndarray}}}{{ $\rightarrow$ ndarray}}
\pysigstopsignatures
\sphinxAtStartPar
Rotates each frame in the given trajectories according to a random quaternion.
\begin{quote}\begin{description}
\sphinxlineitem{Parameters}
\sphinxAtStartPar
\sphinxstyleliteralstrong{\sphinxupquote{trajectories}} (\sphinxstyleliteralemphasis{\sphinxupquote{list of md.trajectory}}) \textendash{} A list of mdtraj.trajectory objects to be modified.

\sphinxlineitem{Returns}
\sphinxAtStartPar
Returns a list of trajectories, the frames of which have been randomly rotated and wrapped back into the box.

\sphinxlineitem{Return type}
\sphinxAtStartPar
list of md.trajectory

\end{description}\end{quote}

\end{fulllineitems}

\index{unsorted\_segment\_sum() (in module NNucleate.utils)@\spxentry{unsorted\_segment\_sum()}\spxextra{in module NNucleate.utils}}

\begin{fulllineitems}
\phantomsection\label{\detokenize{NNucleate:NNucleate.utils.unsorted_segment_sum}}
\pysigstartsignatures
\pysiglinewithargsret{\sphinxcode{\sphinxupquote{NNucleate.utils.}}\sphinxbfcode{\sphinxupquote{unsorted\_segment\_sum}}}{\emph{\DUrole{n}{data}\DUrole{p}{:}\DUrole{w}{  }\DUrole{n}{Tensor}}, \emph{\DUrole{n}{segment\_ids}\DUrole{p}{:}\DUrole{w}{  }\DUrole{n}{Tensor}}, \emph{\DUrole{n}{num\_segments}\DUrole{p}{:}\DUrole{w}{  }\DUrole{n}{int}}}{{ $\rightarrow$ Tensor}}
\pysigstopsignatures
\sphinxAtStartPar
Function that sums the segments of a matrix. Each row has a non\sphinxhyphen{}unique ID and all rows with the same ID are summed such that a matrix with the number of rows equal to the number of unique IDs is obtained.
\begin{quote}\begin{description}
\sphinxlineitem{Parameters}\begin{itemize}
\item {} 
\sphinxAtStartPar
\sphinxstyleliteralstrong{\sphinxupquote{data}} (\sphinxstyleliteralemphasis{\sphinxupquote{torch.tensor}}) \textendash{} A tensor that contains the data that is to be summed.

\item {} 
\sphinxAtStartPar
\sphinxstyleliteralstrong{\sphinxupquote{segment\_ids}} (\sphinxstyleliteralemphasis{\sphinxupquote{torch.tensor}}) \textendash{} An array that has the same number of entries as data has rows which indicates which rows shall be summed.

\item {} 
\sphinxAtStartPar
\sphinxstyleliteralstrong{\sphinxupquote{num\_segments}} (\sphinxstyleliteralemphasis{\sphinxupquote{int}}) \textendash{} This is the number of unique IDs, i.e. the dimensionality of the resulting tensor.

\end{itemize}

\sphinxlineitem{Returns}
\sphinxAtStartPar
Returns a tensor shaped num\_segments x data.size(1) containing all the segment sums.

\sphinxlineitem{Return type}
\sphinxAtStartPar
torch.Tensor

\end{description}\end{quote}

\end{fulllineitems}



\chapter{Module contents}
\label{\detokenize{NNucleate:module-NNucleate}}\label{\detokenize{NNucleate:module-contents}}\index{module@\spxentry{module}!NNucleate@\spxentry{NNucleate}}\index{NNucleate@\spxentry{NNucleate}!module@\spxentry{module}}

\renewcommand{\indexname}{Python Module Index}
\begin{sphinxtheindex}
\let\bigletter\sphinxstyleindexlettergroup
\bigletter{n}
\item\relax\sphinxstyleindexentry{NNucleate}\sphinxstyleindexpageref{NNucleate:\detokenize{module-NNucleate}}
\item\relax\sphinxstyleindexentry{NNucleate.data\_augmentation}\sphinxstyleindexpageref{NNucleate:\detokenize{module-NNucleate.data_augmentation}}
\item\relax\sphinxstyleindexentry{NNucleate.dataset}\sphinxstyleindexpageref{NNucleate:\detokenize{module-NNucleate.dataset}}
\item\relax\sphinxstyleindexentry{NNucleate.models}\sphinxstyleindexpageref{NNucleate:\detokenize{module-NNucleate.models}}
\item\relax\sphinxstyleindexentry{NNucleate.pycv\_link}\sphinxstyleindexpageref{NNucleate:\detokenize{module-NNucleate.pycv_link}}
\item\relax\sphinxstyleindexentry{NNucleate.training}\sphinxstyleindexpageref{NNucleate:\detokenize{module-NNucleate.training}}
\item\relax\sphinxstyleindexentry{NNucleate.utils}\sphinxstyleindexpageref{NNucleate:\detokenize{module-NNucleate.utils}}
\end{sphinxtheindex}

\renewcommand{\indexname}{Index}
\printindex
\end{document}